\section{The abstract ledger model and determinism. }
\label{sec:the-model}
A ledger $L$ is defined by its state type $\State_L$, transaction
type $\Tx_L$, initial state $\fun{initState}_L~:~\type{State}_L$,
and a state update function $ups_L~:~\type{Tx}_L~\to~\type{State}_L~\to~(\type{State}_L)_{\Err}$.

When a $tx$ is not valid in a state $s$, $\ups_L~tx~s~=~\Err$.
A valid ledger state is one that is the result
of applying a sequence of transactions to the initial state that do not
produce $\Err$. This conception of a valid on-chain state implies that any difference between
local and on-chain state is the result of a different sequence of transctions
having been applied in each case.

We present a weaker and a stronger definition of determinism. The stronger definition,
which we call order-determinism (OD), says that
any two permutations of a list of transactions must produce the same state when
applied to a give valid state, unless one (or both) of them produces an $\Err$.
The weaker definition, called update-determinism (UD), requires that if two transactions produce the same state
when applied to a valid state, they must also produce the same state when applied
to any other valid state. We conjecture that the weaker constraint is a consequence of
the stronger one.

\section{Mathematical models of abstract ledgers. }
\label{sec:math}

An important part of future and ongoing determinism research is the application of
mathematical tools and constructs to characterize determinism in ledgers.
We define two categories that model ledgers. One is a one-object category of all valid states
as well as $\Err$, with the maps in the category being specified by the transactions.
The second one is a category of all ledgers, together with all maps between them that preserve the initial
state and update function. We also define a free, finitely generated monoid whose
underlying set, for a given ledger, is all dependent pairs of a lists of transactions
and corresponding states computed by the $\ups$ function.

Theory of changes (ToC) \cite{incremental} is a framework for defining differentiation
on changes to data structures. We adjust this formalism to partial functions so that
it can be employed in the context of ledgers and ledger updates.
A notion of a type representing a change
set is difficult to define while remaining agnostic of the specific data structure, however,
we observe that sets of ledger changes (i.e. $\Diff_L$ type) correspond exactly to transactions,
and $\ups$ is a function that applies those changes.
We use this as a tool for formalizing the colloquial non-determinism definition,
and present a proof that it is equivalent to OD. We then also
use this differentiation formalism to analyze the update-determinism.

So far, our reasoning has been agnostic of the content and structure of ledger
states. In many cases, specific parts of the ledger are of interest for the
study of determinism, such as an account or smart contract state. We introduce
the concept of threads to formalize what we mean by ledger parts, as well
as define what it means for an individual thread to be deterministic.
We show that all threads must be OD in an OD ledger.


\section{Examples. }
\label{sec:examples}

We can apply the tools for ledger analysys discussed above to an existing ledger ---
the Cardano ledger with smart contract integration \cite{alonzo}.
We note that a component, called pointer addresses,
of the ledger is a thread with an UP, but not OD update
procedure. The $\ups$ function, then, uses the ledger data generated during pointer
computation to perform stake distribution calculations require OD of their
components to maintain OD of the ledger as a whole.

On the other hand, smart contracts implementing state machines \cite{eutxo-ma} constitute threads,
which have indeed been shown to be deterministic \cite{alonzo}. Work remains to formally demonstrate
that the two definitions of determinism for smart contracts align.
