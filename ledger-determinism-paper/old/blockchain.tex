\section{Blocks, transactions, and threads}
\label{sec:blockchain}

We can use the notion of threads to represent the relationship between blocks
and transactions.
Let us consider an illustrative example, which we will afterwards generalize to
formally state the structure most blockchains are instances of.
Consider the types are as specified in Figure \ref{fig:block-thread}.

\begin{figure*}[htb]
  \emph{Block example types}
  %
  \begin{equation*}
    \begin{array}{r@{~=~}l@{~~~~}lr}
      \Value
      & \Z
      & \text{The ledger's currency} \\
      \Ix
      & \N
      & \text{Index type} \\
      \TxIn
      & (\Tx, \Ix)
      & \text{Transaction input} \\
      \TxOut
      & \Ix~\mapsto~\Value
      & \text{Transaction outputs} \\
      \UTxO
      & \TxIn~\mapsto~\Value
      & \text{UTxO} \\
      \type{UTxOTx}
      & ([\TxIn],~\TxOut,~\Slot)
      & \text{Transaction} \\
      \type{Block}
      & (\Block^?,~[\type{UTxOTx}])
      & \text{Block} \\
      \type{ChainState}
      & (\Block^?,~\Slot,~\UTxO)
      & \text{Chain state} \\
    \end{array}
  \end{equation*}
  \caption{Types of a block-based ledger}
  \label{fig:block-thread}
\end{figure*}

In blockchains, as the name suggests, the atomic state update
is done by a block rather than a transaction. Blocks contain lists of transactions, as well
as other data used to perform the update.
Block application updates the \emph{chain state}, which the state updated by
transactions is a part of. So, blocks play the role of the $\Tx~\leteq~\Block$ type in our
model, and the $\State$ refers to the chain state.

We define the initial state by $\fun{initState}~\leteq~(\Nt,~0,~\empty)$, and the block application function is

\[ \ups~(b,~lstx)~(b_c,~slot,~utxo)~=~\begin{cases}
  \Err & \text{ if } \not~\fun{blockChecks}~(b,~lstx)~(b_c,~slot,~utxo) \\
  ((b,~lstx),~slot~+~1,~\fun{foldl}~(\fun{updateUTxO}~slot)~utxo~lstx) & \text{ otherwise }
\end{cases} \]

Performs the following checks via the $\fun{blockChecks}$ function, and returns $\Err$
either fails :

\begin{itemize}
  \item[(i)] $b~==~b_c$, which ensures that the (last) block recorded in the state is
  the same block to which the new block specifies that it must be attached

  \item[(ii)] $lstx~\neq~[]$, which ensures that blocks must contain at least one
  transaction

  \item[(ii)] $\fun{foldl}~(\fun{updateUTxO}~slot)~utxo~lstx~\neq~\Err$, which verifies
  that the UTxO set is updated without errors by left-folding $\fun{updateUTxO}$ over
  the list of transactions.
\end{itemize}

The $\fun{updateUTxO}$ update is done in the usual way, by removing all entries corresponding
to transaction inputs, and adding ones corresponding to the outputs. The following
checks are performed, and $\Err$ is returned by $\fun{updateUTxO}~slot~(lsin,~outs,~lastslot)~utxo$
if any of them fail :

\begin{itemize}
  \item[(i)]  $lsin ~\neq~ \Nt$ ensures the transaction has at least one input  \\

  \item[(ii)] ensures the current slot is before or equal to the slot up to which the
  transaction is valid

  \item[(iii)] $\forall i \in lsin, i \mapsto \wcard \in utxo$ ensures all the transaction
  inputs correspond to existing UTxO entries \\

  \item[(iv)] $\not~\Sigma_{\wcard \mapsto v\in outs} v~ =~\Sigma_{v\in \{v~\mid~i \in lsin \wedge  i\mapsto v~\in~utxo \}} v$
  is the preservation of value condition, which ensure that the total value in the consumed UTxO entries
  is equal to the value in the entries produced  \\

\end{itemize}

\[ \fun{updateUTxO}~slot~(lsin,~outs,~lastslot)~utxo~=~
  (utxo~\setminus~\{ i \mapsto \wcard \mid i \in lsin \}) \cup \{ ((lsin,~outs),~ix) \mapsto v \mid ix \mapsto v \in outs  \]

Note here that we use the full block and transaction data where their hashes would
normally be used instead : within keys of UTxO entries, and inside
blocks to specify the preceeding block. The reason for this is that properties
we discuss here may not hold up without the assumption there will be no hash collisions,
and additional machinery is required to reason under this assumption, while using
hashes instead of preimages does not appear to provide additional insight into the
properties we study here. However, a closer examination of this claim could be
part of future work.

We see immediately that the update function of the UTxO set (along with the
constraints on it) in this example can be specified by

\[ \fun{updateUTxOThread} ~lstx ~(\wcard,~slot,~ utxo)~ utxo'~ \leteq~ \begin{cases}
  \fun{foldl}~(\fun{updateUTxO}~slot)~utxo'~lstx & \text { if } utxo~==~utxo' \\
  \Err & \text{ otherwise}
\end{cases}  \]

Since the thread update is only valid if the argument corresponding to the state of the
thread (here, $utxo'$) is equal to the the projection from the chain state to the
thread state (here, $utxo$ in the chain state triple).

The slot number is not used in the generation
of the updated UTxO set in the case when a UTxO set, and not $\Err$, is returned. However, in order to
determine whether a thread update is permitted, the $slot$ value, which
is not included in the UTxO thread data, is used needed.
We will show that it is in fact a constraint on the design of deterministic ledgers
that this pattern of thread update
be followed, ie. the updated state of a thread can depend on data external to the thread,
but only in so far that the update can produce an error for certain values if that data.

We can make this example more general by remaining agnostic of the data types of the
chain state. Following this example,

\begin{definition}
  \begin{itemize}
    \item[(i)] Given a type $\Tx$, a $\Block$ is a type which admits
    the following accessors :

    \[ \fun{prevBlock} \in \Block \to \Block \]

    \[ \fun{txList} \in~\Block \to [\Tx] \]

%    \[ \fun{time} \in \type{Header} \to \Time \]

    \item[(ii)] A $\type{ChainState}$ is a type with the following accessors

    \[ \fun{lastBlock} \in \type{ChainState}~\to \Block \]

    \[ \fun{txState} \in \Block \to \type{TxState}  \]

    where $\type{TxState}$.

    \item[(iv)] Given an initial state $\fun{initState}~\in~\Block$ and an update procedure
    $\ups~\in~\Block~\to~\type{ChainState}~\to~\type{ChainState}$, the tuple
    $(\Block,~\type{ChainState},~\ups,~\fun{initState})$ is said to be a
    \emph{blockchain} if

    \begin{itemize}
      \item[(1)] it is a ledger specification,

      \item[(2)] it admits a thread $\type{TxState}$ with the update function defined by

     \[ \fun{updateLedger} ~lstx ~(led,~ os)~ led'~ \leteq~ \begin{cases}
       \fun{foldl}~(\fun{updateLedger}~os)~led'~lstx & \text { if } led~==~led' \\
       \Err & \text{ otherwise}
     \end{cases}  \]

    \end{itemize}

     and

    wherein $(\fun{updateLedger}~os)$

    A  $(\Block,~\type{ChainState},~\fun{initState})
    and an update procedure


  \end{itemize}

  , as we will see in the next definition, represents the
  part of the $\type{ChainState}$ that applying transactions updates

\end{definition}

Recall here, again, that our formalism is not probabilistic, so, unlike a realistic
deployed blockchain, it does not rely on hash functions having a low probability
of collisions. Instead, we use pre-images where hashes would normally be used in
practice (eg. as pointers to previous blocks).

we will refer to the \emph{ledger} as the thread in the
chain state that can be described as \emph{any data updatable by a transaction}. Formally,

While the above specification is a fairly abstract idea of what a blockchain is,
some distributed-consensus append-only ledgers may have a structure which deviates from the
specification we present here. However, the
\cite{Cardano-ledger-spec} \cite{tezos} \cite{ethereum} \cite{Nakamoto}



In the case of updating the UTxO set and the protocol parameter record, this
situation can manifest itself as, eg. a specifying a UTxO update which disallows some
new UTxOs to be added to it if they do not satisfy a constraint that depends on
one of the protocol parameters. There might be, for instance, a "minimum value constraint"
imposed by the $\fun{updateT}_{UTxO}~tx$ function
on any new UTxO being added by a $tx$ such that the value contained in a UTxO entry must be greater
than a value indicated in some protocol parameter in $\fun{minV}$. The update specification
for protocol paramenters may also have some constraints.

Any transaction that is valid (ie. does not produce an $\Err$) must satisfy both the
constraints on the UTxO update, as well as those for the protocol parameter updates.
