\section{Introduction}
\label{sec:intro}

In the context of blockchain transaction processing and smart contract execution,
determinism is usually taken to mean something like "the ability to predict locally,
before submitting a transaction, the on-chain result of processing that transaction and its scripts".
This is an important aspect of ledger design because users care about being able to accurately predict
fees they will be charged, rewards they will receive from staking, outcomes of
smart contract executions, etc. before submitting transactions. The purpose of this
work is to formalize this property of ledgers, and study the constraints under
which it can be guaranteed, thereby providing analysis tools and design principles for building ledgers
whose transaction processing outcomes can be accurately forecast.

Blockchain ledger and consensus design relies on the definition of the transaction processing function
itself being entirely deterministic. The impossibility of predicting
the exact on-chain state transactions will be applied to, however, is due to
unpredictable network propagation of transactions, resulting in an arbitrary
order in which they are processed as the source of non-determinism. Determinism
can, therefore, be formulated entirely in the language of transaction application commutativity,
but we retain the commonly used blockchain terminology in this work.

We present an abstract
ledger model capturing the architectural core shared by most blockchain platforms:
a ledger is a state transition system, with valid transactions (or blocks) as the only transitions.
We then formalize the definition of determinism in terms of this abstract functional
specification of ledger structure, and use mathematical tools
for analyzing them in order is to establish a way to reason about
conditions under which the transaction order has no effect on the resulting state
or parts thereof.

A similar construction to our ledger definition is presented in \cite{paxos}.
However, the focus there is on the ordering of state updates
to optimize the execution of a particular consensus algorithm.
Another related study of commutativity of state transitions is ~\cite{commautomata}.
