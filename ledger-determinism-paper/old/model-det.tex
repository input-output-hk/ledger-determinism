\section{The abstract ledger model and determinism}
\label{sec:the-model}
A ledger is defined by its state type $\State$, transaction
type $\Tx$, initial state $\fun{initState} : \State$,
and a state update function $\ups : \Tx \to \State \to \State \uplus \bot$.

A transaction $tx$ is called \emph{valid} in state $s$ whenever $\ups\ tx\ s \neq \bot$.
A valid ledger state is one that is the result
of applying a sequence of transactions to the initial state, where each transaction
is valid in the state it is applied to. We call such a sequence a \emph{trace}.

Defining valid state in this way allows us to make precise the nature of the
discrepancy between a valid local and a valid on-chain state that may prevent a
user from accurately predicting the consequences of their transaction being applied :
since both states are the result of applying a valid sequence of transactions,
the only reason they may be distinct is that their traces differ (usually because
some more recent transactions have not yet been applied to the local state).

We present two definitions of determinism and establish a relation between them. The first definition,
which we call \textbf{order-determinism}, requires that
any two permutations of a list of transactions must produce the same state when
applied to a given valid state, unless one (or both) of them produces an $\bot$.
A ledger where the $\State$ is the collection of all multisets of $\Tx$,
and $\ups~tx~s \leteq s \uplus \{tx\}$, is
an example of an order-deterministic ledger.

The second definition, called \textbf{update-determinism}, requires that if two
transactions produce the same state
when applied to a valid state, they must also produce the same state when applied
to any other valid state. An example of a
ledger that is neither is one with $\State = \Tx = \B$,
and an update function that flips the boolean in the state if the transaction and
state booleans match.


\section{Mathematical models of abstract ledgers}
\label{sec:math}

An important part of future and ongoing determinism research is the application of
mathematical tools and constructs to characterize determinism in ledgers,
such as category theory and group theory.
We define a one-object category of all valid states plus $\bot$
where the maps are specified by the transactions, as well as
two other related categories.
We also define a category of all ledgers, together with all maps between them that preserve the initial
state and update function.
We also define a free, finitely generated monoid whose underlying set, for a given ledger,
consists of all dependent pairs of a list of transactions and the corresponding states computed by the $\ups$ function.

The \emph{theory of changes}  is a framework for defining differentiation
of functions specifying updates to data structures~\cite{incremental}. We adjust this formalism to partial functions so that
it can be employed in the context of ledgers and ledger updates.
The notion of a type representing a change
set is difficult to define while remaining agnostic of the specific data structure, however,
we observe that every permissible set of ledger changes corresponds exactly
to a sequence of valid transactions, and $\ups$ is the function that applies those changes.
We apply this idea to formalize the colloquial determinism definition,
and present a proof that the definition of determinism we give in terms of
change sets and derivatives is equivalent to order-determinism. We then also
use this differentiation formalism to analyze update-determinism and relate
it to order-determinism.

So far, our reasoning has been agnostic of the content and structure of ledger
states. In many cases, specific parts of the ledger are of interest for the
study of determinism, such as an account or smart contract state. We introduce
the concept of threads to formalize what we mean by ledger parts, as well
as define what it means for an individual thread to be deterministic.
We show that all threads must be order-deterministic in an order-deterministic ledger.
