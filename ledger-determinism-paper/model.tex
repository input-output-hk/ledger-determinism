\section{The Abstract Ledger Model}
\label{sec:model}

We introduce our ledger model and define some basic structure.

\todopv{make notes of assumptions of finiteness or at least classical logic?}

\subsection{What is a ledger?}
\label{sec:what}

A ledger specification is given by the types and functions in Figure \ref{fig:ledger-spec}.
We use the notation $\odot$ to suggest right monoid action of transaction lists on states.
The monoid operation on lists is concatenation, which is compatible with the state
update function $\odot$.


\begin{figure*}[htb]
  \emph{Ledger Specification Types}
  %
  \begin{equation*}
    \begin{array}{r@{~\in~}l@{~~~~}lr}
      \Tx
      & \Set
      & \text{The type of the transactions} \\
      \State
      & \Set
      & \text{The type of the ledger state} \\
    \end{array}
  \end{equation*}
  \emph{Ledger Specification Functions}
  %
  \begin{equation*}
    \begin{array}{r@{~\in~}l@{~~~~}lr}
    \ups & ~\type{State} \to \type{Tx} \to \type{State}
    & \text{Updates state with a given transaction} \\
    \fun{initState} & ~\type{State}
    & \text{The initial state of the specification} \\
    \end{array}
  \end{equation*}
  \emph{Ledger Specification Notation}
  %
  \begin{equation*}
    \begin{array}{r@{~}l@{~~~~}lr}
    s~\odot~tx~ ~=~&~ \ups~s~tx  \\
    & \text{Update state $s$ with transaction $tx$} \\
    \nextdef
    s~\odot~[tx_1, ... , tx_n]~~= &~ \ups~(...~ (\ups~(\ups~s~tx_1)~tx_2) ~...)~ tx_n \\
    & \text{Update state $s$ with transaction $tx$} \\
    \end{array}
  \end{equation*}
  \emph{Compatibility}
  %
  \begin{equation*}
    \begin{array}{r@{}l@{~~~~}lr}
    \fun{respectsConcat}~:~&~ \forall s, txs1, txs2,~~s \odot txs1 \odot txs2 = s \odot (txs1 ++ txs2)  \\
    & \text{List concatenation is compatible with $\odot$} \\
    \end{array}
  \end{equation*}
  \caption{Ledger specification types and functions}
  \label{fig:ledger-spec}
\end{figure*}

\subsection{Error states}
\label{sec:errors}

Given a ledger specification, we can define the following additional structure,
see Figure \ref{fig:ledger-aux}. A ledger may have a collection of \emph{error states} $\Err$,
each of which having only one possible transition out of it - back to itself.
We will sometimes abuse notation by writing $s = \Err$ instead of $s \in \Err$.

The collection of \emph{valid states} $\V$ contains all states reachable from the initial state,
including any error states, if those are reachable. Note here that we abuse the
notation $s \odot tx$, by using it for both $s \in \State$ and $s \in \V$.


\begin{figure*}[htb]
    \emph{Error States and Valid States}
    %
    \begin{align*}
      & \Err ~\in~ \powerset{\State} \\
      & \Err ~=~ \{~s~\mid~\forall~tx,~\ups~s~tx~=~s~\} \\
      & \text{Collection of error states} \\
      %
      \nextdef
      & \V ~\in~ \Set \\
      & \V ~=~
          \{ ~s~ \mid ~\exists~\var{lstx}~\in~[Tx],~\fun{initState}~\odot~\var{lstx}~=~s~\} \\
      & \text{Set of all valid states} \\
    \end{align*}
  \caption{Ledger specification additional structure}
  \label{fig:ledger-aux}
\end{figure*}

\pagebreak

\subsection{Threads}
\label{sec:threads}

We call a \emph{thread} in a ledger the pair $(\T, \pi, e)$ of a target type and a function
from the ledger state to $\T$, as specified in Figure \ref{fig:ledger-thread}.
Threads will be used to study the evolution of substates
of the ledger. We can consider these as "observables" within the ledger state.

A thread $(\T, \pi, e)$ induces a ledger to which we refer
to as a $\T$-subledger of the original ledger. To construct
this subledger, we specify its transaction type as a pair of a transaction $tx$
and the larger state $s$
to which it is being applied, denoted by $tx_s$. We define the
update function and initial state of the $T$-subledger in terms of data in the
bigger ledger, as in Figure \ref{fig:ledger-thread}.


\begin{figure*}[htb]
  \emph{Threads structure}
  %
  \begin{align*}
    & \T ~\in~ \Set \\
    & \text{The type of a thread} \\
    \nextdef
    & \pi ~\in~ \State \to \T \\
    & \text{Function computing the value of a ledger thread at a given state} \\
    \nextdef
    & e ~\in~ \{~\pi~ s ~\mid~ s \in \Err~\} \\
    & \text{The error that occurs when the update cannot be applied because of mismatched state} \\
  \end{align*}
  \emph{Thread subledger}
  %
  \begin{align*}
    & \fun{initState}_{\T} \in \T \\
    & \fun{initState}_{\T} = \pi ~\fun{initState} \\
    & \text{Initial state of the subledger defined by the thread T} \\
    \nextdef
    & \odot_T \in \T (\State \times \Tx) \to \T \\
    & q~\odot_T~(s, tx) = \begin{cases}
      e & \text{ if } q \neq (\pi~ s) \\
      \pi~(s \odot tx) \text{ otherwise}
    \end{cases} \\
    & \text{Update function of a T-ledger} \\
  \end{align*}
  \emph{Thread properties}
  %
  \begin{align*}
    & \fun{sameErrs} ~=~ \forall~s \in \V, s \in \Err \Rightarrow \pi~s \in \Err \\
    & \text{Any subledger specified by a thread enforces mapping errors to errors} \\
  \end{align*}
  \caption{Ledger threads}
  \label{fig:ledger-thread}
\end{figure*}

\pagebreak

\subsection{Ledger and state comparison relations}
\label{sec:relations}

To reason about ledger determinism, we would like to be able to specify the difference between
and to compare valid states. We begin by re-stating our ledger update function as
a tertiary relation rather than a two-argument function. Let us define
such a relation $\mathcal{LR}$, see Figure \ref{fig:ledger-aux}.

We would like to describe arbitrary paths between any two valid non-$\Err$ states.
For this we define the relation $\mathcal{UR}$, which is similar to $\mathcal{LR}$, but each
of the transactions in the transition across a sequence valid states can appear
either in the forward direction $s_i \odot tx = s_{i+1}$ (specified as $(s_i, [(\True, tx)], s_{i+1})$),
or in the backward direction, $s_{i+1} \odot tx = s_{i}$, specified as $(s_i, [(\False, tx)], s_{})$.

We use the following notation for representing triples that are in the $\mathcal{UR}$ relation :

\[ \langle s, utxs, s' \rangle ~~ \Rightarrow ~~(s, txs, s') \in  \mathcal{UR} \]

We introduce the notation $\boxdot$ in Figure \ref{fig:ledger-aux}, which allows us to
perform an

We define the \emph{delta} of two valid ledger states, denoted $\Delta~(s, s')$, to be the collection of paths
from $s$ to $s'$ in the undirected graph formed by valid, non-$\Err$ states as vertices and edges labelled
by pairs of a transaction transitioning between the two states, and the direction in which
it goes, expressed as a boolean.
This collection is never empty, since $[s, ..., \fun{initState},..., s'] \in \Delta~(s, s')$.
In some sense, the collection of fall paths between
them. Note here that distinct transactions (and transaction sequences)
may give the same state update, so that $s \odot txs = s' \odot txs$ for some distinct $s, s'$.

We introduce the relation $*=$ to compare states. Two states satisfy this relation
if they are equal, or one of them is an $\Err$ state. Note that this relation is
not transitive, as if $s \neq s'$, it still holds that $s *= \Err *= s'$.

We introducte the relation $=*$ to compare collections of transition sequences out of two
valid states. Two collections are $=*$-equal whenever they both contain the same
sequences of transitions, except for possibly those sequences in either that lead
to an $\Err$ state. Note that this relation is also not transitive, since

\todopv{example}

\begin{figure*}[htb]
  \begin{align*}
    & \boxdot  ~\in~ \State \to (\Bool,~\Tx) \to \powerset{\State} \\
    & s \boxdot (b,~tx) ~=~ \{~s'~\mid~~b \Rightarrow s~\odot~tx~=~s'~ \wedge \neg b \Rightarrow s'~\odot~tx~=~s\} \\
    & \text{Notation of applying or unapplying a transaction} \\
    \nextdef
    & \mathcal{LR}  ~\in~ \powerset (\State \times [\Tx] \times \State) \\
    & \mathcal{LR} = \{~ (s \in \V, txs \in [\Tx], s' \in \V)~\mid~\Err \neq s\odot txs = s' \neq \Err \} \\
    & \text{All non-error transitions between valid ledger states} \\
    \nextdef
    & \mathcal{UR}  ~\in~ \powerset (\State \times [(\Bool, \Tx)] \times \State) \\
    & \mathcal{UR} =\{~(s, [], s')~\mid~\Err\neq s = s'\} ~\cup~ \{~ (s \in \V, [(b_1, tx_1), ... , (b_k, tx_k)] \in [\Tx], s' \in \V)~\mid~\\
    & ~~~~ \exists~ [s_1, ..., s_{k+1}],~s = s_1 \neq \Err,~s' = s_{k+1} \neq \Err, ~k \geq 1, \forall 1 \leq i \leq k,~  \\
    & ~~~~ ((b_i \wedge (s_{i}, [tx_i], s_{i+1}) \in \mathcal{LR}) \vee (\neg b_i \wedge (s_{i+1}, [tx_i], s_{i})~\in~\mathcal{LR})~\} \\
    & \text{All non-error transitions in either direction between valid ledger states} \\
    \nextdef
    & \boxdot  ~\in~ \State \to [\Bool,~\Tx] \to \powerset{\State} \\
    & s \boxdot [(b_1,~tx_1), ..., (b_k,~tx_k)] ~=~ (s \boxdot ) \\
    & \text{Notation of applying or unapplying a transaction} \\
    \nextdef
    & \Delta ~\in~ (\V \times \V) \to \powerset{[(\Bool, \Tx)]} \\
    & \Delta (s, s')~=~\{~utxs~\mid~ (s, utxs, s') \in \mathcal{UR}~\} \\
    & \text{Collection of paths from $s$ to $s'$ in two-directional transaction graph} \\
    %
    \nextdef
    & (*=) ~\in~ \State \to \State \to \Bool \\
    & s~ *=~ s' ~=~ (s = s') \vee (s \in \Err) \vee (s' \in \Err) \\
    & \text{Relation containing pairs of identical states or a state and an error state} \\
    %
    \nextdef
    & (=*) ~\in~ \V \to \V \to \powerset{[(\Bool, \Tx)]} \to \powerset{[(\Bool, \Tx)]} \to \Bool \\
    & \var{stx1} ~=*_{s1,s2}~ \var{stx2} ~=~
     (\forall~\var{lstx}~\in~\var{stx1},~s1\boxdot\var{lstx}~\subseteq~\Err~\vee~\var{lstx}~\in~\var{stx2}) \\
    & ~~~~ \wedge~ (\forall~\var{lstx}~\in~\var{stx2},~s2\boxdot\var{lstx}~\subseteq~\Err~\vee~\var{lstx}~\in~\var{stx1}) \\
    & \text{Relation compating sets of transitions from two different states} \\
  \end{align*}
  \caption{Comparisons of states and state transition collections}
  \label{fig:ledger-aux}
\end{figure*}
