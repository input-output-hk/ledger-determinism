\section{Deterministic Ledgers}
\label{sec:determinism}

Here we give definitions of determinism and compare them.

\subsection{Thread-determinism}
\label{sec:thread-det}

We first introduce the most intuinitve concept of determinism in Figure \ref{fig:thread-det}.
A ledger specification in which the $\fun{threadDeterminism}$ contraint is satisfied
guarantees that the update of any substate of the ledger is independent of
other data on the ledger external to the substate. That is, when a substate in a (valid)
ledger state $s$ is updated by some transaction, it will be updated
to the same substate in any state $s'$ where the original substate is the same as in $s$.

We say a single thread $(\T, \pi)$ is deterministic whenever $\fun{threadDeterminism}~\T~\pi$
holds.

\todopv{this is the kind of determinism that we use in showing deterministic script validation}

\textbf{Proposition. } See the \emph{Thread-determinism proposition for transaction sequences}
in Figure \ref{fig:thread-det}.

\textbf{Proof.} By induction.

\begin{itemize}
  \item[(i)] Base case, $txs = []$ :
  \[\pi~s~=~\pi~s'~~\Rightarrow~~\pi~s~=~\pi~s' \]
  \item[(ii)] Inductive step for $txs ++ tx$ : we assume that,
  \[ \pi~s~=~\pi~s',~\pi~(s\odot txs)~*=~\pi~(s'\odot txs) \]

  Then, by $\fun{threadDeterminism}$ at states $\pi~(s\odot txs)$ and $\pi~(s'\odot txs)$,

  we conclude the result,
  \[ \pi~(s\odot (txs ++ tx)) ~=~\pi~((s\odot txs) \odot tx) ~*=~\pi~((s'\odot txs) \odot tx)~=~\pi~(s'\odot (txs ++ tx)) \]
\end{itemize}

\begin{figure*}[htb]
  \emph{Thread-determinism definition on a ledger specification}
  %
  \begin{equation*}
    \begin{array}{l@{~~}l@{~~}ll}
    \fun{threadDeterminism} ~=~ & \forall~\T,~\pi~\in~\State \to \T, \\
    & ~~~~ \forall~(s,~s'~\in~\V),~t~\in~\T,~tx \in \Tx,~\pi~s~=~\pi~s', \\
    & ~~~~ \pi~(s\odot tx)~*=~\pi~(s'\odot tx)
    \end{array}
  \end{equation*}
  \emph{Thread-determinism proposition for transaction sequences}
  %
  \begin{equation*}
    \begin{array}{l@{~~}l@{~~}ll}
    \fun{threadDeterminism} ~=~ & \forall~\T,~\pi~\in~\State \to \T, \\
    & ~~~~ \forall~(s,~s'~\in~\V),~t~\in~\T,~txs \in [\Tx],~\pi~s~=~\pi~s', \\
    & ~~~~ \pi~(s\odot txs)~*=~\pi~(s'\odot txs)~\\
    \end{array}
  \end{equation*}
  \caption{Thread-determinism}
  \label{fig:thread-det}
\end{figure*}

\subsection{Delta-determinism}
\label{sec:delta-det}

Delta-determinism is a constraint on a ledger specitication that guarantees that
the changes made by a transaction sequence $txs$ to some state $s$, expressed as
the collection of non-$\Err$ transactions that all perform the same state update,
are the same as the changes made by $txs$ to any other state $s'$.

This definition of determinism quantifies over all paths in the transition
graph, making proving this property directly intractable.

\begin{figure*}[htb]
  \emph{State-determinism constraint on a ledger specification}
  %
  \begin{equation*}
    \begin{array}{l@{~~}l@{~~}ll}
    \fun{deltaDeterminism} ~=~ & \forall (s~s'~\in~\V)~(txs~\in~[\Tx]),~\\
    & ~~~~ \Delta (s,~s\odot txs)~*=_{s,s'}~\Delta (s',~s'\odot txs)
    \end{array}
  \end{equation*}
  \caption{State-determinism}
  \label{fig:delta-det}
\end{figure*}

\subsection{Order-determinism}
\label{sec:order-det}

The definition in Figure \ref{fig:order-det} is given in terms of the traces leading
to a particular ledger state. Specifically, it is concerned with the independence of the
ledger state from the order
of the transactions in the trace that has lead to any given state.

\begin{figure*}[htb]
  \emph{Order-determinism constraint on a ledger specification}
  %
  \begin{equation*}
    \begin{array}{l@{~~}l@{~~}ll}
    \fun{orderDeterminism} ~=~ & \forall~(s~\in~\V)~(txs~\in~[\Tx]),~\\
    & ~~~~txs'~\in~\fun{Permutation}~txs,~\\
    & ~~~~s\odot txs~\neq~\Err~\neq~s \odot~txs' \\
    & ~~~~\Rightarrow~~s\odot txs~=~s \odot~txs'
    \end{array}
  \end{equation*}
  \caption{Order-determinism}
  \label{fig:order-det}
\end{figure*}


\subsection{Aspirational results}

\begin{itemize}
  \item[(i)] State and thread det's are equivalent
  \item[(ii)] Both imply order-det
  \item[(iii)] Injectivity (questionable, probably at least needs additional preconditions) : \\
  $\forall s, s', txs, s \odot txs = s' \odot txs \Rightarrow s = s'$)
\end{itemize}

% order => update?
% update 0 A  =  A
%  update 0 B  =  B
%  update 0 C  =  C
%  update 0 X  =  err
%
%  update 1 A  =  A
%  update 1 B  =  C
%  update 1 C  =  B
%  update 1 X  =  err
%
%  update a A  =  err
%  update a B  =  err
%  update a C  =  err
%  update a X  =  A
%
%  update b A  =  err
%  update b B  =  err
%  update b C  =  err
%  update b X  =  B
