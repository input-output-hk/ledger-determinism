\section{Introduction}
\label{sec:intro}

In the context of blockchain transaction processing and smart contract execution,
determinism is usually taken to mean something like "the ability to predict locally,
before submitting a transaction, the on-chain result of processing that transaction and its scripts".
This is an important aspect of ledger design because users care about being able to accurately predict
fees they will be charged, rewards they will receive from staking, outcomes of
smart contract executions, etc. before submitting transactions. The purpose of this
work is to formalize this property of ledgers, and study the constraints under
which it can be guaranteed, thereby providing analysis tools and design principles for building ledgers
whose transaction processing outcomes can be accurately forecast.

Blockchain ledger and consensus design relies on the property that
replaying a its transaction/block history produces
the same output every time. We refer to this property as \emph{historic determinism}.
The focus of in this work is \emph{predictive determinism}, which focuses on
reasoning about the changes a transaction makes, and when these changes
are themselves predictable, without needing to know the state to which they
will be applied.

The difficulty in predicting
the exact on-chain state to which transactions/blocks will be applied is due to
unpredictable network propagation of transactions, consensus design,
rollbacks, malicious actors, etc. Full predictive determinism, which we will henceforth
just call determinism, guarantees if a transaction
is valid, the variations in the state due to the reasons mentioned above will not
affect how the transaction is processed.

In this work, we first specify a ledger model, then give some definitions of what
it means for it to be deterministic. We also consider real-world constraints
on ledger design that prevent full determinism from being implementable, and
give an analysis of how to treat such less-deterministic ledgers in a useful way.

Previous work on predictive determinism, with a similar ledger construction
as we present here, is found in \cite{parallelism}. There are notable differences,
such as a fixed set of observables, no analysis of transaction failure or analysis
of determinism for the cases of dropped transactions, and the focus is more
on use of existing ledgers in a deterministic way, rather than ledger design
principles.
A similar construction to our ledger definition is presented in \cite{paxos}.
However, the focus there is on the ordering of state updates
to optimize the execution of a particular consensus algorithm.
Another related study of commutativity of state transitions is ~\cite{commautomata}.
