\section{Categories of Ledgers}
\label{sec:cats}

There are different categories one may construct out of a ledger specification.

\textbf{Proposition : Valid states as pullbacks in $\Set$. } Suppose
$\mathcal{P}$ is a pullback in $\Set$, \newline

\begin{center}
  \xymatrix{
      \mathcal{P} \pullbackcorner \ar[r]^{\pi_2} \ar[d]_{\pi_1}   & [\Tx] \ar[d]^{\fun{initState} \odot \wcard} \\
      \State \ar[r]^{\fun{id}_{\State}}                         & \State
  }
\end{center}

or a given ledger $L = (\State, \Tx, \odot, \fun{initState})$. Then the set of all
valid states $\V = \pi_2 \mathcal{P}$.

\textbf{Proof. } Recall that

\[ \V~=~ \{ ~s~ \mid ~\exists~\var{lstx}~\in~[Tx],~\fun{initState}~\odot~\var{lstx}~=~s~\} \]

Recall that a pullback in $\Set$ is defined by

\[ \mathcal{P}~=~ \{ ~(txs, s)~ \mid ~\fun{initState}~\odot~\var{txs}~=~s~\} \]

Now, for any $s \in \V$, we pick a $txs$ such that $\fun{initState}~\odot~\var{txs}~=~s$,
since such a $txs$ exists. So, $(txs, s) \in \mathcal{P} ~~\Rightarrow ~~s \in \pi_2 \mathcal{P}$.

For any $s \in \pi_2 \mathcal{P}$, there exists $(txs, s) \in \mathcal{P}$ such that
$\pi_2 (txs, s) = s$, and $\fun{initState}~\odot~\var{txs}~=~s$, which proves the result.

\subsection{Single-ledger category}
\label{sec:slc}

Given a ledger $L = (\State, \Tx, \odot, \fun{initState})$, we define the category $\type{SL}$, constructed
from a single ledger specification's types, transitions, and initial state,
in the following way :

\begin{itemize}
  \item[(i)] Objects : $\fun{Obj}~\type{SL} = \State$ \newline

  \item[(ii)] Morphisms : $\fun{Hom}~S1~S2 = \{ ~\wcard~\odot~txs \mid ~txs \in [\Tx]~,~S1~ \odot~txs~=~S2 \}$ \newline

  \item[(iii)] Identity : $\fun{id}_{S1} = [] ~\in~\fun{Hom}~S1~S1 $ \newline

  \item[(iv)] Composition : $\forall~txs1,~txs2, S$ \\
  $(\wcard~\odot~txs2) ~\circ~(\wcard~\odot~txs1) (S) = (\wcard~\odot~txs2) (S~\odot~txs1) = (S~\odot~txs1) \odot txs2  \\
  ~~~~ =  S~\odot~(txs1 ++ txs2)~~\Rightarrow ~~(\wcard~\odot~(txs1 ++ txs2))~\in ~\fun{Hom}~S1~S3$

  \item[(v)] Associativity : holds because the monoid action respects list concatenation
  in the list monoid (ie. $\fun{respectsConcat}$)\\

  % $\forall~txs1,~txs2,~tx3, S$ \\
  % $(\wcard~\odot~txs3) ~\circ~((\wcard~\odot~txs2) ~\circ~(\wcard~\odot~txs1)) (S)
  % = (\wcard~\odot~txs3) ((\wcard~\odot~txs2) (\wcard~\odot~txs1) (S)) \\
  % = (\wcard~\odot~txs3) ((\wcard~\odot~txs2) (S~\odot~txs1)) =  (\wcard~\odot~txs3) (S~\odot~(txs1 ++ txs2)) \\
  % =  (S~\odot~(txs1 ++ txs2)) \odot txs3 = (S~\odot~((txs1 ++ txs2) ++ txs3))
  %
  %  ~\fun{Hom}~S1~S3$
\end{itemize}


\subsection{Category of all ledgers}
\label{sec:all-ledgers}

We first introduce some notions needed to describe the category of all ledgers $\mathcal{L}$
and maps between them that preserve ledger structure.

Let $\Set_{*}$ denote the category of pointed sets with maps $f \in \fun{Hom} (\{*\} \to S,~\{*\} \to Q)$,
such that $({*} \mapsto x \in S, s \in S) \mapsto (* \mapsto f(x), f(s)) \in Q$ for some
set map $f : S \to Q$.

Let $\Monoid$ denote a category of monoids $(M : \Set,~ \mu~:~ \Set \times \Set \to \Set,~ e : M)$
and monoid action- and identity-preserving homomorphisms. We use $(M, \mu, e)$ and $M$
interchangeably here.

Let $\langle \wcard \rangle ~:~\Monoid~\dashv ~\Set~ :~ \mathcal{U}$ be the free-forgetful adjunction
for monoids.

Let us now define a functor $\Gamma$ :

\[ \Gamma : \begin{cases}
  \Set_{*} \times \Monoid \to \Set_{*} \times \Monoid  \\
  (i : \{*\} \to S, M) \mapsto (\{*\} \to S \times \mathcal{U} M, M) \\
  (f \times g : (i : \{*\} \to S, M) \to (\{*\} \to Q, N)) \mapsto
    ((\langle f,~\mathcal{U}~g \rangle, g)  (\{*\} \to (S \times \mathcal{U} M), M) \to (\{*\} \to (Q \times \mathcal{U} N), N))
\end{cases} \]

We now define the natural transformations $\gamma : \Gamma^2 \to \Gamma$ and
$\eta : \mathbb{1}_{\Set \times \Monoid} \to \Gamma$.

Note that

\[ \Gamma (i : \{*\} \to S, M) = (\langle i, \mathcal{U} e\rangle : \{*\} \to S \times \mathcal{U} M, M) \]

and

\[\Gamma^2 (i : \{*\} \to S, M) =
(\langle i, \mathcal{U} e, \mathcal{U} e\rangle : \{*\} \to S \times \mathcal{U} M \times \mathcal{U} M, M) \]

So that we define

\[ \gamma_{(i : \{*\} \to S, M)} : \begin{cases}
   \Gamma^2 (i : \{*\} \to S, M) \to \Gamma (i : \{*\} \to S, M)
  \\
  (* \mapsto (i, \mathcal{U} e, \mathcal{U} e), (s, m_1, m_2), m) \mapsto (* \mapsto (i, \mathcal{U} (\mu e e)), (s, \mathcal{U} (\mu m_1 m_2), m)
\end{cases} \]

\[ \eta_{(i : \{*\} \to S, M)} : \begin{cases}
  (i : \{*\} \to S, M) \to (\langle i, \mathcal{U} e\rangle : \{*\} \to S \times \mathcal{U} M, M)  \\
  (* \mapsto i, s, m) \mapsto (* \mapsto (i, \mathcal{U} e), (s, \mathcal{U} e) m)
\end{cases} \]

The data $(\Gamma, \gamma, \eta)$ is a monad on $\Set_{*} \times \Monoid$. A $\Gamma$-algebra
is made up of $(i : \{*\} \to S, M)$ together with the following operation, induced
by a monoid action $\odot : S \times M \to S$

\[ \odot_{(i : \{*\} \to S, M)} : \begin{cases}
  \Gamma (i : \{*\} \to S, M) \to (i : \{*\} \to S, M) \\
  (* \mapsto (i, \mathcal{U} e), (s, m_1), m) \mapsto (* \mapsto i, (s \odot m_1), m)
\end{cases} \]

We define the category of all ledgers, $\mathcal{L}$, in terms of these structures.
Recall here that the morphisms in the product category with objects $\Set_{*} \times \Monoid$
are those that preserve $*$ in the first coordinate, and monoid structure in the second.

\begin{itemize}
  \item[(i)] Objects : \newline
  $\fun{Obj}~\mathcal{L} = \Set_{*} \times \Monoid$ \newline

  \item[(ii)] Morphisms are a subset of the morphisms of the product morphisms
  between objects of $\Set_{*} \times \Monoid$ which contains only those maps that
  preserve each $\odot_{(i : \{*\} \to S, M)}$  : \newline

  $\fun{Hom}~(i : \{*\} \to S, M)~(i : \{*\} \to Q, N) = \{ f~:~(i : \{*\} \to S, M) \to (i : \{*\} \to Q, N) ~\mid~\\
  ~~~~ \odot_{(i : \{*\} \to Q, N)}~\circ~(\Gamma~ f)~=~ f \circ \odot_{(i : \{*\} \to S, M)} \}$ \newline

  \item[(iii)] Identity :
  $ \odot_{(i : \{*\} \to Q, N)}~\circ~(\Gamma~ \mathbb{1}_{(i : \{*\} \to S, M)})~ \\
  ~~~~~ = \odot_{(i : \{*\} \to Q, N)}~\circ~ \mathbb{1}_{\Gamma~(i : \{*\} \to S, M)} \\
  ~~~~~ = \odot_{(i : \{*\} \to Q, N)} \\
  ~~~~~ = \mathbb{1}_{(i : \{*\} \to S, M)} \circ \odot_{(i : \{*\} \to Q, N)}$ \newline

  \item[(iv)] Composition (we drop the subscript of $\odot$ from here onwards) : \\
  $\forall~f \in \fun{Hom}~(i : \{*\} \to S, M)~(i : \{*\} \to Q, N),~
  g \in \fun{Hom}~(i : \{*\} \to Q, N)~(i : \{*\} \to P, R), \\
  (g \circ f)~\circ~\odot~ = ~g (f \circ \odot) = ~g ~(\odot \circ (\Gamma f)) \\
  ~~~~ = ~g ~(\odot \circ (\Gamma f)) \\
  ~~~~ = ~\odot \circ (\Gamma g) \circ (\Gamma f) \\
  ~~~~ = ~\odot \circ (\Gamma (g \circ f))$ \newline

  \item[(v)] Associativity : inherited from maps in $\Set_{*} \times \Monoid$.

\end{itemize}
