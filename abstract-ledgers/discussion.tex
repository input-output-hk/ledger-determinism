\section{Discussion}

\subparagraph*{Assumptions and limitations}
We use the term "transaction" to refer to the ledger state transition type
for the reason that we make the assumption that most users are concerned with
application of transactions to the ledger state.
However, in practice, an atomic update of a blockchain's state is a block,
which updates ledger data that transaction application cannot modify,
such as the hash of the previous block or the current slot number.
The notion of threads lets us formalize the relation between
block-based and transaction-based updates, but we leave this for future work.

Our model is constructed to reflect the simplifying
assumption that there is exactly one way to interface with a ledger --- by applying
a transaction, which is indeed the case in most normal circumstances.
Our model is also strictly functional, so that the update function itself
is necessarily deterministic. We also make the assumption that the update function
does not evolve in any way.

\vspace{-.2cm}
\subparagraph*{Applications to Cardano.}
We can apply the tools for ledger analysis discussed above to an existing ledger ---
the Cardano ledger with smart contract integration~\cite{alonzo}.
We note that a component of the ledger, called pointer addresses,
is a thread which is update-deterministic, but not order-deterministic.

On the other hand, we conjecture that smart contracts implemented as state machines~\cite{eutxo} constitute threads.
Smart contract validation has been shown to be deterministic according to a
notion of determinism specific to the Cardano ledger presented in ~\cite{alonzo}.
Work remains to formally demonstrate that contracts do indeed constitute threads,
and that that the definition of determinism in ~\cite{alonzo} is equivalent
to the order-determinism we present in this work.

\vspace{-.2cm}
\subparagraph*{Future work}
As part of future and ongoing work, we intend to continue using the mathematical
tools we discussed here to investigate
ledger determinism. We
hope to formulate a local, as opposed to trace-based, characterization of determinism,
as well as characterize transaction validity in deterministic ledgers.

% Other future work includes describing and identifying properties of ledger states
% which can be expressed as partial categorical products of substates, and the related
% notion of threads.
% In the category of all ledgers, the structure and properties of morphisms is also of interest.
% Another future research direction is the further study of differentiation
% of transaction application in terms of how imposing various
% constraints on derivatives affects the determinism of ledger updates.
% Finally, we also plan to use group-theoretic
% tactics to further study the structure of the ledger monoid.
