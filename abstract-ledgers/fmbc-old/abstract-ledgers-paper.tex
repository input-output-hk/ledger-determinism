
\documentclass[a4paper,UKenglish,cleveref, autoref, thm-restate]{lipics-v2021}
%This is a template for producing LIPIcs articles.
%See lipics-v2021-authors-guidelines.pdf for further information.
%for A4 paper format use option "a4paper", for US-letter use option "letterpaper"
%for british hyphenation rules use option "UKenglish", for american hyphenation rules use option "USenglish"
%for section-numbered lemmas etc., use "numberwithinsect"
%for enabling cleveref support, use "cleveref"
%for enabling autoref support, use "autoref"
%for anonymousing the authors (e.g. for double-blind review), add "anonymous"
%for enabling thm-restate support, use "thm-restate"
%for enabling a two-column layout for the author/affilation part (only applicable for > 6 authors), use "authorcolumns"
%for producing a PDF according the PDF/A standard, add "pdfa"

%\pdfoutput=1 %uncomment to ensure pdflatex processing (mandatatory e.g. to submit to arXiv)
%\hideLIPIcs  %uncomment to remove references to LIPIcs series (logo, DOI, ...), e.g. when preparing a pre-final version to be uploaded to arXiv or another public repository

%\graphicspath{{./graphics/}}%helpful if your graphic files are in another directory

\bibliographystyle{plainurl}% the mandatory bibstyle

\title{Determinism of ledger updates} %TODO Please add

%\titlerunning{Dummy short title} %TODO optional, please use if title is longer than one line

\author{James	Chapman}{IOHK} {james.chapman@iohk.io}{}{}

\author{Andre	Knispel}{IOHK} {andre.knispel@iohk.io}{}{}

\author{Orestis	Melkonian}{IOHK \and University of Edinburgh, UK} {orestis.melkonian@ed.ac.uk}{}{}

\author{Polina	Vinogradova}{IOHK} {polina.vinogradova@iohk.io}{}{}

%
% \author{Jane {Open Access}}{Dummy University Computing Laboratory, [optional: Address], Country \and My second affiliation, Country \and \url{http://www.myhomepage.edu} }{johnqpublic@dummyuni.org}{https://orcid.org/0000-0002-1825-0097}{(Optional) author-specific funding acknowledgements}%TODO mandatory, please use full name; only 1 author per \author macro; first two parameters are mandatory, other parameters can be empty. Please provide at least the name of the affiliation and the country. The full address is optional. Use additional curly braces to indicate the correct name splitting when the last name consists of multiple name parts.

\authorrunning{J.	Chapman and A.	Knispel and O.	Melkonian and P.	Vinogradova}
 %TODO mandatory. First: Use abbreviated first/middle names. Second (only in severe cases): Use first author plus 'et al.'

\Copyright{James	Chapman and Andre	Knispel and Orestis	Melkonian and Polina	Vinogradova}
%TODO mandatory, please use full first names. LIPIcs license is "CC-BY";  http://creativecommons.org/licenses/by/3.0/
% Polina	Vinogradova	polina.vinogradova@iohk.io	Canada	IOHK		✔
% Orestis	Melkonian	orestis.melkonian@ed.ac.uk	United Kingdom	University of Edinburgh, IOHK		✔
% Andre	Knispel	andre.knispel@iohk.io	Germany	IOHK		✔
% James	Chapman	james.chapman@iohk.io	United Kingdom	IOHK

\begin{CCSXML}
<ccs2012>
<concept>
<concept_id>10003752.10003766</concept_id>
<concept_desc>Theory of computation~Formal languages and automata theory</concept_desc>
<concept_significance>500</concept_significance>
</concept>
</ccs2012>
\end{CCSXML}

\ccsdesc[100]{Theory of computation~Formal languages and automata theory}
%\ccsdesc[100]{\textcolor{red}{Replace ccsdesc macro with valid one}} %TODO mandatory: Please choose ACM 2012 classifications from https://dl.acm.org/ccs/ccs_flat.cfm

\keywords{determinism, ledger model, theory of changes} %TODO mandatory; please add comma-separated list of keywords

\category{extended abstract} %optional, e.g. invited paper

\relatedversion{} %optional, e.g. full version hosted on arXiv, HAL, or other respository/website
%\relatedversiondetails[linktext={opt. text shown instead of the URL}, cite=DBLP:books/mk/GrayR93]{Classification (e.g. Full Version, Extended Version, Previous Version}{URL to related version} %linktext and cite are optional

%\supplement{}%optional, e.g. related research data, source code, ... hosted on a repository like zenodo, figshare, GitHub, ...
%\supplementdetails[linktext={opt. text shown instead of the URL}, cite=DBLP:books/mk/GrayR93, subcategory={Description, Subcategory}, swhid={Software Heritage Identifier}]{General Classification (e.g. Software, Dataset, Model, ...)}{URL to related version} %linktext, cite, and subcategory are optional

%\funding{(Optional) general funding statement \dots}%optional, to capture a funding statement, which applies to all authors. Please enter author specific funding statements as fifth argument of the \author macro.



%\nolinenumbers %uncomment to disable line numbering



%Editor-only macros:: begin (do not touch as author)%%%%%%%%%%%%%%%%%%%%%%%%%%%%%%%%%%
\EventEditors{John Q. Open and Joan R. Access}
\EventNoEds{2}
\EventLongTitle{42nd Conference on Very Important Topics (CVIT 2016)}
\EventShortTitle{CVIT 2016}
\EventAcronym{CVIT}
\EventYear{2016}
\EventDate{December 24--27, 2016}
\EventLocation{Little Whinging, United Kingdom}
\EventLogo{}
\SeriesVolume{42}
\ArticleNo{23}
%%%%%%%%%%%%%%%%%%%%%%%%%%%%%%%%%%%%%%%%%%%%%%%%%%%%%%
% PACKAGES %

%\usepackage{natbib}
\usepackage{url}

% *** MATHS PACKAGES ***
%
\usepackage{etoolbox}
\usepackage{tikz-qtree}

%\usepackage[cmex10]{amsmath}
%\usepackage{amssymb}
%\usepackage{stmaryrd}
%\usepackage{amsthm}

% \usepackage[margin=2.5cm]{geometry}
\usepackage{iohk}
\usepackage{microtype}
\usepackage{mathpazo} % nice fonts
%\usepackage{amsmath}
\usepackage{amssymb}
%\usepackage{amsthm}
\usepackage{latexsym}
\usepackage{mathtools}
\usepackage{stmaryrd}
\usepackage{extarrows}
\usepackage{slashed}
%\usepackage[unicode=true,pdftex,pdfa,colorlinks=true]{hyperref}
%\usepackage{xcolor}
%\usepackage[capitalise,noabbrev,nameinlink]{cleveref}
%\usepackage{float}

% *** ALIGNMENT PACKAGES ***
%
\usepackage{array}
%\usepackage{float}  %% Try to improve placement of figures.  Doesn't work well with subcaption package.
% \usepackage{subcaption}
\usepackage{caption}

% \usepackage{subfiles}
% \usepackage{geometry}
% \usepackage{listings}
% % \usepackage[dvipsnames]{xcolor}
% \usepackage{verbatim}
% \usepackage{listings}% http://ctan.org/pkg/listings
% \lstset{
%   basicstyle=\ttfamily,
%   mathescape
% }
% \usepackage{alltt}
% \usepackage{paralist}

\usepackage{todonotes}

% COMMANDS %


\newcommand{\code}{\texttt}
\renewcommand{\i}{\textit}  % Just to speed up typing: replace these in the final version
\renewcommand{\t}{\texttt}  % Just to speed up typing: replace these in the final version
\newcommand{\s}{\textsf}  % Just to speed up typing: replace these in the final version
\newcommand{\msf}[1]{\ensuremath{\mathsf{#1}}}
\newcommand{\mi}[1]{\ensuremath{\mathit{#1}}}

%% A figure with rules above and below.
\newcommand\rfskip{3pt}
%\newenvironment{ruledfigure}[1]{\begin{figure}[#1]\hrule\vspace{\rfskip}}{\vspace{\rfskip}\hrule\end{figure}}
\newenvironment{ruledfigure}[1]{\begin{figure}[#1]}{\end{figure}}

%% Various text macros
\newcommand{\true}{\textsf{true}}
\newcommand{\false}{\textsf{false}}

\newcommand{\hash}[1]{\ensuremath{#1^{\#}}}

\newcommand\mapsTo{\ensuremath{\mapsto}}
\newcommand\cL{\ensuremath{\{}}
\newcommand\cR{\ensuremath{\}}}

\newcommand{\List}[1]{\ensuremath{\s{List}[#1]}}
\newcommand{\Set}[1]{\ensuremath{\s{Set}[#1]}}
\newcommand{\FinSet}[1]{\ensuremath{\s{FinSet}[#1]}}
\newcommand{\Interval}[1]{\ensuremath{\s{Interval}[#1]}}
\newcommand{\FinSup}[2]{\ensuremath{\s{FinSup}[#1,\linebreak[0]#2]}}
% ^ \linebeak is to avoid a bad line break when we talk about finite
% maps.  We may be able to remove it in the final version.
\newcommand{\supp}{\msf{supp}}

\newcommand{\Script}{\ensuremath{\s{Script}}}
\newcommand{\scriptAddr}{\msf{scriptAddr}}
\newcommand{\ctx}{\ensuremath{\s{Context}}}
\newcommand{\vlctx}{\ensuremath{\s{ValidatorContext}}}
\newcommand{\mpsctx}{\ensuremath{\s{PolicyContext}}}
\newcommand{\toData}{\ensuremath{\s{toData}}}
\newcommand{\toTxData}{\ensuremath{\s{toTxData}}}
\newcommand{\fromData}{\msf{fromData}}

\newcommand{\emptymap}{\ensuremath{\{\}}}
\newcommand{\verify}{\msf{verify}}

\newcommand{\mkContext}{\ensuremath{\s{mkContext}}}
\newcommand{\mkVlContext}{\ensuremath{\s{mkValidatorContext}}}
\newcommand{\mkMpsContext}{\ensuremath{\s{mkPolicyContext}}}
\newcommand{\checkSig}{\ensuremath{\s{checkSig}}}

\newcommand{\applyScript}[1]{\ensuremath{\llbracket#1\rrbracket}}
\newcommand{\applyMPScript}[1]{\ensuremath{\llbracket#1\rrbracket}}

% Macros for eutxo things.
\newcommand{\tx}{\mi{tx}}
\newcommand{\TxId}{\ensuremath{\s{TxId}}}
\newcommand{\txId}{\msf{txId}}
\newcommand{\txrefid}{\mi{id}}
\newcommand{\Address}{\ensuremath{\s{Address}}}
\newcommand{\DataHash}{\ensuremath{\s{DataHash}}}
\newcommand{\hashData}{\msf{dataHash}}
\newcommand{\idx}{\mi{index}}
\newcommand{\inputs}{\mi{inputs}}
\newcommand{\outputs}{\mi{outputs}}
\newcommand{\validityInterval}{\mi{validityInterval}}
\newcommand{\scripts}{\mi{scripts}}
\newcommand{\forge}{\mi{forge}}
\newcommand{\forgeScripts}{\mi{forgeScripts}}
\newcommand{\sigs}{\mi{sigs}}
\newcommand{\fee}{\mi{fee}}
\newcommand{\addr}{\mi{addr}}
\newcommand{\pubkey}{\mi{pubkey}}
\newcommand{\val}{\type{Value}}  %% \value is already defined

\newcommand{\validator}{\mi{validator}}
\newcommand{\redeemer}{\mi{redeemer}}
\newcommand{\datum}{\mi{datum}}
\newcommand{\datumHash}{\mi{datumHash}}
\newcommand{\datumWits}{\mi{datumWitnesses}}
\newcommand{\Data}{\ensuremath{\s{Data}}}
\newcommand{\Input}{\ensuremath{\s{Input}}}
\newcommand{\Output}{\ensuremath{\s{Output}}}
\newcommand{\OutputRef}{\ensuremath{\s{OutputRef}}}
\newcommand{\Signature}{\ensuremath{\s{Signature}}}
\newcommand{\Ledger}{\ensuremath{\s{Ledger}}}

\newcommand{\outputref}{\mi{outputRef}}
\newcommand{\outputrefs}{\mi{outputRefs}}
\newcommand{\txin}{\mi{in}}
\newcommand{\id}{\mi{id}}
\newcommand{\lookupTx}{\msf{lookupTx}}
\newcommand{\getSpent}{\msf{getSpentOutput}}

\newcommand{\Tick}{\ensuremath{\s{Tick}}}
\newcommand{\currentTick}{\msf{currentTick}}
\newcommand{\spent}{\msf{spentOutputs}}
\newcommand{\unspent}{\msf{unspentOutputs}}
\newcommand{\txunspent}{\msf{unspentTxOutputs}}
\newcommand{\eutxotx}{\msf{Tx}}


\newcommand{\consumes}[1]{\msf{consumes(#1)}}
\newcommand{\consumesOne}[1]{\msf{consumesOne(#1)}}
\newcommand{\cid}{\mi{cid}}
\newcommand{\inputValue}{\mi{inputValue}}
\newcommand{\rMin}{r_{\mi{min}}}
\newcommand{\rMax}{r_{\mi{max}}}

\newcommand{\utxotx}{\msf{Tx}}

\newcommand{\Quantity}{\ensuremath{\s{Quantity}}}
\newcommand{\Asset}{\ensuremath{\s{Asset}}}
\newcommand{\Policy}{\ensuremath{\s{PolicyID}}}
\newcommand{\Quantities}{\ensuremath{\s{Quantities}}}
\newcommand{\nativeCur}{\ensuremath{\mathrm{nativeC}}}
\newcommand{\nativeTok}{\ensuremath{\mathrm{nativeT}}}
\newcommand{\valC}{\mkValidator{\mathcal{C}}}
\newcommand{\polC}{\mkPolicy{\mathcal{C}}}
\newcommand\mkValidator[1]{\msf{validator}_#1}
\newcommand\mkPolicy[1]{\msf{policy}_#1}

\newcommand{\PublicKey}{\ensuremath{\s{PubKey}}}
\newcommand{\PrivateKey}{\ensuremath{\s{PrivateKey}}}

\newcommand{\pkey}{\ensuremath{\pi_{\mathsf{p}}}}
\newcommand{\skey}{\ensuremath{\pi_{\mathsf{s}}}}

\newcommand\B{\ensuremath{\mathbb{B}}}
\newcommand\N{\ensuremath{\mathbb{N}}}
\newcommand\Z{\ensuremath{\mathbb{Z}}}
\renewcommand\H{\ensuremath{\mathbb{H}}}
%% \H is usually the Hungarian double acute accent
\newcommand{\emptyBs}{\ensuremath{\emptyset}}
\newcommand{\leteq}{\ensuremath{\mathrel{\mathop:}=}}
\newcommand{\Nt}{\ensuremath{\Diamond}}
\newcommand{\Bool}{\type{Bool}}
\newcommand{\Type}{\type{Type}}
\newcommand{\Diff}{\type{Diff}}
\newcommand{\ValSt}{\type{ValSt}}

%ledger spec commands
\newcommand{\Request}{\type{Request}}
\newcommand{\LS}{\mathcal{L}\mathcal{S}}
\newcommand{\State}{\type{State}}
\newcommand{\AccState}{\type{AccState}}
\newcommand{\accid}{\type{AccID}}
\newcommand{\PCMT}{\type{PCMT}}
\newcommand{\Accts}{\type{Accts}}
\newcommand{\Open}{\type{Open}}
\newcommand{\OArgs}{\type{OArgs}}
\newcommand{\Close}{\type{Close}}
\newcommand{\CArgs}{\type{CArgs}}
\newcommand{\Deposit}{\type{Deposit}}
\newcommand{\DArgs}{\type{DArgs}}
\newcommand{\Withdraw}{\type{Withdraw}}
\newcommand{\WArgs}{\type{WArgs}}
\newcommand{\Transfer}{\type{Transfer}}
\newcommand{\TArgs}{\type{TArgs}}
\newcommand{\AccInput}{\type{AccInput}}
\newcommand{\Tx}{\type{Tx}}
\newcommand{\ID}{\type{ID}}
\newcommand{\Th}{\mathcal{T}}
\newcommand{\Tu}{\mathcal{U}}
\newcommand{\T}{\type{T}}
\newcommand{\Err}{\type{Err}}
\newcommand{\ups}{\fun{update}}
\newcommand{\FHBMT}{\type{FHBMT}}
\newcommand{\TxInfo}{\type{TxInfo}}

\newcommand{\True}{\type{True}}
\newcommand{\False}{\type{False}}


% For anonymisation
\newtoggle{anonymous}
\toggletrue{anonymous}
\iftoggle{anonymous}{
  \newcommand{\Cardano}{CHAIN}
  \newcommand{\Plutus}{LANG}
}{
  \newcommand{\Cardano}{Cardano}
  \newcommand{\Plutus}{Plutus Core}
}

% Names, for consistency
\newcommand{\UTXO}{UTXO}
\newcommand{\EUTXO}{E\UTXO{}}
\newcommand{\ExUTXO}{Extended \UTXO{}}
\newcommand{\CEM}{CEM}
\newcommand{\UTXOma}{\UTXO$_{\textrm{ma}}$}
\newcommand{\EUTXOma}{\EUTXO$_{\textrm{ma}}$}

\newcommand\initial{\msf{initial}}
\newcommand\nft{\blacklozenge}
\newcommand\step{\msf{step}}
\newcommand\satisfies{\msf{satisfies}}
\newcommand\checkOutputs{\msf{checkOutputs}}
\newcommand\txeq{tx^\equiv}

% relaxed float placement
\renewcommand{\topfraction}{.95}
\renewcommand{\bottomfraction}{.7}
\renewcommand{\textfraction}{.15}
\renewcommand{\floatpagefraction}{.66}
\renewcommand{\dbltopfraction}{.66}
\renewcommand{\dblfloatpagefraction}{.66}
\setcounter{topnumber}{9}
\setcounter{bottomnumber}{9}
\setcounter{totalnumber}{20}
\setcounter{dbltopnumber}{9}

\newcommand\CStep[1]{\ensuremath{
 #1 \xrightarrow{\hspace{5pt} i \hspace{5pt}} (#1' , \txeq)
%%\textsf{step}\, #1\, i \equiv \textsf{just}\, #1'
}}

\begin{document}

\maketitle

%TODO mandatory: add short abstract of the document
\begin{abstract}
  Ensuring deterministic behaviour in distributed blockchain ledger design matters to
  end users because it allows for locally predictable fees, smart contract evaluation
  outcomes, and updates to other ledger-tracked data. In this work we begin by defining
  an abstract interface of ledgers and its update procedure, which gives us the ability
  to make formal comparisons of different ledger designs across various properties. We
  use this model as a basis for formalizing and studying several properties colloquially
  classified as determinism of ledgers.

  We identify a stronger and a weaker definition of determinism, providing simple but
  illustrative examples. We situate both versions of determinism in the context of the
  theory of changes, and conjecture what constraints on the derivation of ledger update
  functions are sufficient and necessary for the two definitions. We additionally discuss
  substates of a ledger state, which we refer to as threads, and outline how particular
  threads can remain deterministic while the full ledger may not be.

  We discuss how these ideas can be applied to realistic ledgers' designs and architectures,
  and analyze a nuanced example of non-determinism in an existing UTxO ledger with the
  tools we have developed.

\end{abstract}

\section{Introduction}
\label{sec:intro}

In the context of blockchain transaction processing and smart contract execution,
determinism is usually taken to mean something like "the ability to predict locally,
before submitting a transaction, the on-chain result of processing that transaction and its scripts".
This is an important aspect of ledger design because users care about being able to accurately predict
fees they will be charged, rewards they will receive from staking, outcomes of
smart contract executions, etc. before submitting transactions. The purpose of this
work is to formalize this property of ledgers, and study the constraints under
which it can be guaranteed, thereby providing analysis tools and design principles for building ledgers
whose transaction processing outcomes can be accurately forecast.

Blockchain ledger and consensus design relies on the definition of the transaction processing function
itself being entirely deterministic. The impossibility of predicting
the exact on-chain state transactions will be applied to, however, is due to
unpredictable network propagation of transactions, resulting in an arbitrary
order in which they are processed as the source of non-determinism. Determinism
can, therefore, be formulated entirely in the language of transaction application commutativity,
but we retain the commonly used blockchain terminology in this work.

We present an abstract
ledger model capturing the architectural core shared by most blockchain platforms:
a ledger is a state transition system, with valid transactions (or blocks) as the only transitions.
We then formalize the definition of determinism in terms of this abstract functional
specification of ledger structure, and use mathematical tools
for analyzing them in order is to establish a way to reason about
conditions under which the transaction order has no effect on the resulting state
or parts thereof.

A similar construction to our ledger definition is presented in \cite{paxos}.
However, the focus there is on the ordering of state updates
to optimize the execution of a particular consensus algorithm.
Another related study of commutativity of state transitions is ~\cite{commautomata}.

\section{The abstract ledger model and determinism}
\label{sec:the-model}
A ledger is defined by its state type $\State$, transaction
type $\Tx$, initial state $\fun{initState} : \State$,
and a state update function $\ups : \Tx \to \State \to \State \uplus \bot$.

A transaction $tx$ is called \emph{valid} in state $s$ whenever $\ups\ tx\ s \neq \bot$.
A valid ledger state is one that is the result
of applying a sequence of transactions to the initial state, where each transaction
is valid in the state it is applied to. We call such a sequence a \emph{trace}.

Defining valid state in this way allows us to make precise the nature of the
discrepancy between a valid local and a valid on-chain state that may prevent a
user from accurately predicting the consequences of their transaction being applied :
since both states are the result of applying a valid sequence of transactions,
the only reason they may be distinct is that their traces differ (usually because
some more recent transactions have not yet been applied to the local state).

We present two definitions of determinism and establish a relation between them. The first definition,
which we call \textbf{order-determinism}, requires that
any two permutations of a list of transactions must produce the same state when
applied to a given valid state, unless one (or both) of them produces an $\bot$.
A ledger where the $\State$ is the collection of all multisets of $\Tx$,
and $\ups~tx~s \leteq s \uplus \{tx\}$, is
an example of an order-deterministic ledger.

The second definition, called \textbf{update-determinism}, requires that if two
transactions produce the same state
when applied to a valid state, they must also produce the same state when applied
to any other valid state. An example of a
ledger that is neither is one with $\State = \Tx = \B$,
and an update function that flips the boolean in the state if the transaction and
state booleans match.


\section{Mathematical models of abstract ledgers}
\label{sec:math}

An important part of future and ongoing determinism research is the application of
mathematical tools and constructs to characterize determinism in ledgers,
such as category theory and group theory.
We define a one-object category of all valid states plus $\bot$
where the maps are specified by the transactions, as well as
two other related categories.
We also define a category of all ledgers, together with all maps between them that preserve the initial
state and update function.
We also define a free, finitely generated monoid whose underlying set, for a given ledger,
consists of all dependent pairs of a list of transactions and the corresponding states computed by the $\ups$ function.

The \emph{theory of changes}  is a framework for defining differentiation
of functions specifying updates to data structures~\cite{incremental}. We adjust this formalism to partial functions so that
it can be employed in the context of ledgers and ledger updates.
The notion of a type representing a change
set is difficult to define while remaining agnostic of the specific data structure, however,
we observe that every permissible set of ledger changes corresponds exactly
to a sequence of valid transactions, and $\ups$ is the function that applies those changes.
We apply this idea to formalize the colloquial determinism definition,
and present a proof that the definition of determinism we give in terms of
change sets and derivatives is equivalent to order-determinism. We then also
use this differentiation formalism to analyze update-determinism and relate
it to order-determinism.

So far, our reasoning has been agnostic of the content and structure of ledger
states. In many cases, specific parts of the ledger are of interest for the
study of determinism, such as an account or smart contract state. We introduce
the concept of threads to formalize what we mean by ledger parts, as well
as define what it means for an individual thread to be deterministic.
We show that all threads must be order-deterministic in an order-deterministic ledger.

\section{Assumptions, simplifications, and limitations. }
We present a ledger API that $\Tx$ to denote the state transitions
of the system for the reason that we make the assumption that most users are concerned with
application of transactions to the ledger state. However, in practice,
an atomic update of a blockchain's state is a block. Blocks update state data
that transactions often do not inspect, such as the hash of the previous block.
The notion of threads lets us formalize the relation between
block-based and transactions-based updates. The implications of this statement,
as well as the special properties of block application, which will be part of future work.

Our abstract model is constructed as an API which reflects the simplifying
assumption that there is exactly one way to interface with a ledger --- by applying
a transition of the transition type. This is the case for most users in most circumstances of normal
blockchain operation. Note here that our model is strictly functional, so that we
assume that the program processing
and applying transactions (blocks) to the ledger does not itself
exhibit non-deterministic behaviour. We also make the assumption that the update function
does not evolve in any way.

\section{Future work. }
As part of future and ongoing work, we intend to continue using mathematical
tools for investigating structure relevant to the study of
ledger and determinism thereof. In particular,
we focus on describing and identifying properties of ledger states
which can be expressed as partial categorical products of substates, and the related
notion of threads. In the category of all ledgers, properties
and structure of the maps are also of interest.
Another future research direction is the further study of differentiation
of transaction application, in terms of the implications for determinism of the different
constraints on derivatives. Finally, we also plan to use the group-theoretic
tactics to further study the structure of the ledger monoid.

%%
%% Bibliography
%%

%% Please use bibtex,

\bibliography{abstract-ledgers}

\appendix


\end{document}
