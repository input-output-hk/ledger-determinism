\section{Assumptions, simplifications, and limitations. }
We present a ledger API that $\Tx$ to denote the state transitions
of the system for the reason that we make the assumption that most users are concerned with
application of transactions to the ledger state. However, in practice,
an atomic update of a blockchain's state is a block. Blocks update state data
that transactions often do not inspect, such as the hash of the previous block.
The notion of threads lets us formalize the relation between
block-based and transactions-based updates. The implications of this statement,
as well as the special properties of block application, which will be part of future work.

Our abstract model is constructed as an API which reflects the simplifying
assumption that there is exactly one way to interface with a ledger --- by applying
a transition of the transition type. This is the case for most users in most circumstances of normal
blockchain operation. Note here that our model is strictly functional, so that we
assume that the program processing
and applying transactions (blocks) to the ledger does not itself
exhibit non-deterministic behaviour. We also make the assumption that the update function
does not evolve in any way.

\section{Future work. }
As part of future and ongoing work, we intend to continue using mathematical
tools for investigating structure relevant to the study of
ledger and determinism thereof. In particular,
we focus on describing and identifying properties of ledger states
which can be expressed as partial categorical products of substates, and the related
notion of threads. In the category of all ledgers, properties
and structure of the maps are also of interest.
Another future research direction is the further study of differentiation
of transaction application, in terms of the implications for determinism of the different
constraints on derivatives. Finally, we also plan to use the group-theoretic
tactics to further study the structure of the ledger monoid.
