\section{Introduction}
\label{sec:intro}

In the context of blockchain transaction processing and smart contract execution,
determinism is usually taken to mean something like "the ability to predict locally,
before subimitting a transaction, the on-chain result of processing that transaction, including its scripts".
This is an important aspect of ledger design because users care about being able to accurately predict
fees they will be charged, rewards they will receive from staking, outcomes of
smart contract executions, etc. before posting transactions.
Transaction processing and smart contract execution on a variety of ledgers can
be informally characterized as either more or less determnistic according to this conception
of determinism.

In order to formalize a shared definition of determinism that can be used to characterize a variety
of ledgers across different platformss, we first present an abstract
ledger model capturing the architectural core shared by a number of prominent
platforms \cite{Cardano-ledger-spec} \cite{tezos} \cite{ethereum} \cite{Nakamoto}.
This core can be summarized as follows : a ledger is a state transition system,
with transactions (or blocks) as the only transitions, some of which are
invalid.

With this simplified functional view of ledgers, we formalize several interpretations
of the informal definition of determinism, and use several mathematical tools
for analysis of it. We argue that
the only source of non-determinism in our model is the impossibility of
predicting the exact on-chain state users' transaction will be applied to,
which is related to commutativity of state transitions (for previous work on
commuting automata, see \cite{commautomata}). This
could be due to unpredictable network propagation of transactions, or malicious
actions on the parts of other users.


%
% \begin{itemize}
%
%   \item[(i)] define what is an abstract ledger UI,
%
%   \item[(ii)] cite as examples the following ledgers implementing it
%   \begin{itemize}
%     \item cardano (https://allquantor.at/blockchainbib/pdf/corduan2019formal.pdf , see Figure 13 for UTxO state update with a transaction, for example),
%     \item bitcoin (i only saw this in the Ethereum paper under "Bitcoin as a state transition system", https://ethereum.org/en/whitepaper/), and
%     \item ethereum (same link as above, see "Ethereum state transition function")
%   \end{itemize}
%
%   \item[(iii)] define order determinism (reordering transactions gives the same state or $\Err$)
%   \item[(iv)] define update determinism (this is a recent change to the paper, i am not sure if this is something you looked at, basically, if for some $s, tx, tx', (s ~tx) = (s ~tx')$ and not $\Err$ , it follows that for all $s', (s' ~tx) = (s' ~tx')$ )
%
%   \item[(v)] argue that update determinism is weaker (i think i can prove this), but still useful
%
%   \item[(vi)] give simple examples of both
%
%   \item[(vii)] discuss a specific example of indeterminism in Cardano (pointer addresses), and explain how this problem is a result of making some of the ledger only update-deterministic, then using this ledger data in calculations that require order determinism
%
%   \item[(viii)] conjecture that an update function that looks like $\pi_{TxS} ~:~ (\Tx,~\State) \to \State$ makes for deterministic ledgers - because projections commute
%
%   \item[(ix)] argue that the UTxO update procedure fits this update model - if you treat a key-value map as a (finite) product of values $(v1, ..., vk)$ corresponding to each of the k possible keys of the type TxId , some of which are $\Nt$ . The projection into $\UTXO$ from the pair $(\Tx, \UTXO)$ implicitly contains to-be-excluded  values, which are specified by their input keys, and new key values to which we project (the outputs)
%
%   \item[(x)] distinguish the UTxO determinism from its replay protection, compare to attested accouts model \\ \\
%
%   and possibly,
%   \item[(xi)]  explain what the constraints on the derivatives (in the sense of incremental lambda calculus) of ledger updates are for (1) order determinism (2) state determinism, which are different
%
%   \item[(xii)]  explain that threads are projections together with their update functions (or "pointers" to specific parts of a ledger, eg. a state machine or the UTxO set in a EUTxO ledger)
%
%   \item[(xiii)]  say that a specific thread can be deterministic even if the whole ledger is not, again giving example of Cardano (state machines are deterministic, point to proof in the ledger spec)
%
% \end{itemize}
