\section{Update-deterministic ledgers}
\label{sec:update}

Determinism, both state- and order- (SD and OD, for short), specifies what it means for a ledger to behave
predictably under an update caused by a transaction in the face of a lack of information about
what the state might be at any given time, or what transactions happen to have
been processed before the transaction doing the update in question.
This approach to specifying determinism focuses on allowing only certain transactions
to be processed.

Here we propose another angle from which to investigate determinism,
which instead focuses on the properties of the $\ups$ function itself, as well
as the state and transaction structure.

A ledger $L$ is \emph{without inverses} whenever

  \[ \forall~lts~s,~(s~\var{lts})~=~s~\Leftrightarrow~\var{lts}~=~[] \]

If a ledger has replay protection, it is without inverses.
Replay protection precludes inverses, since having
inverses implies that applying a list of transactions is possible more than once,

\[ (s~\var{lts}~\var{lts})~=~(s~\var{lts})~=~s \]

A ledger $L$ is \emph{update-deterministic} (or, UD) whenever it has
\emph{consistent transaction application}, defined in Figure \ref{fig:consistent}

\begin{figure*}[htb]
  \begin{equation*}
    \begin{array}{l@{~~}l@{~~}ll}
    \fun{updateDetConstraint}_L ~=~ & \forall~s~tx~tx',~\\
    & ~~~~ (\Err~\neq~(s~\var{tx})~=~(s~\var{tx'})~\neq~\Err~\Rightarrow~\\
    & ~~~~\forall~s',~(s'~\var{tx})~\neq~\Err~\neq~(s'~\var{tx'}) \\
    & ~~~~~~~~\Rightarrow~(s'~\var{tx})~=~(s'~\var{tx'}))
    \end{array}
  \end{equation*}
  \caption{Consistent transaction application}
  \label{fig:consistent}
\end{figure*}


\begin{theorem}
  If a ledger $L$ is order-deterministic, it is update-deterministic.
\end{theorem}

\begin{proof}
   \todopv{Prove this! I think it's true}
\end{proof}

UD is a weaker condition. It captures the idea that while the transaction order
cannot be changed, there are no additional transaction application computations done
using the order (see example (i) and (ii) below). In a sense, some order-dependent
internal indexing takes place when a transaction is applied. If this indexing is not
later inspected by other computations on-chain, there is no issue. However, building
applications than do inspect this indexing data off-chain is bad.

This definition is also representative of the idea that state-determinism attempted to
capture (but ended up defaulting to OD), that a list of transactions makes the same changes to the ledger no matter
what state they are applied to.

\todopv{Give example of pointer addresses on Cardano. The process of building the map
of pointer addresses as transactions update the delegations map could itself be
UD (it is just an internal indexing process), but we notice that in fact, reward
calculations are done using this pointer address
map, so UD is broken.}

\textbf{Examples : } \\

\begin{itemize}
  \item[(i)] $\Tx~ = ~\N$, $\State~ = ~[\N]$, and the update function appends the number in a $tx$ to the list. This ledger
  is update-deterministic.

  \item[(ii)] $\Tx~ = ~\N$, $\State~ = ~{~ n~\in~\N~}^*$, where the update function includes the number in the $tx$ in the multiset
  (${...}^*$ is the notation for multiset). This ledger is OD

  \item[(iii)] $\Tx~ = ~\Bool$, $\State~ = ~\Bool$, where the $tx$ flips the bit in the state if they match. This ledger is not
  deterministic.

  \item[(iv)] $\Tx~ = ~\Bool$, $\State~ = ~\Bool$, where the update function is XOR. This ledger is OD (and so, probably UD),
  but has inverses.
\end{itemize}

\subsection{UD and derivatives}

Recall that in Section \ref{sec:od-sd-d} we discussed the relation between derivation
in the ledger and OD. We can also make a claims about UD and derivation.

\textbf{Claim : } A derivative of a UD ledger must be independent of the state,
but can still depend on the set of changes $ds$ applied before or after the
transaction list $txs$, of which we are calculating the derivative.
Recall the constraint on valid derivatives (using shorthand notation) :

\[ \fun{derivativeConstraint} ~\in~ \forall~s~ds~txs, \]

\[ ~((s~ds)~txs)~\neq~\Err~\neq~((s~txs) ~(txs ~s~ ds)')~\Rightarrow~((s~ds)~txs)~=~((s~txs) ~(txs ~s~ ds)') \]

The $\fun{updateDetConstraint}_L$ then tells us that

\[ \forall~s_1~s_2,~ (s~(txs~++~(txs~s_1~ds)'))~=~(s~(txs~++~(txs~s_2~ds)')) \]

And therefore the derivative $(txs~s~ds)'$ must not be dependent on $s$.

\todopv{can we say anything about this in the other direction? ie. derivative is
independed of s implies ... ?}


\subsection{Implementation of an OD $\ups$ function}

\todopv{clean this up!!}

This section we consider only ledgers that can be represented as partial products, ie.
products $P~\times~Q$ that have functions into them from some type $R$

\[ \langle~\fun{p},~\fun{q}~\rangle~:~R~\to~P~\times~Q \]

return $\Err$ whenever \emph{either} $\fun{p}$ \emph{or} $\fun{q}$ returns $\Err$,
as well as possibly in other cases.

For example, the UTxO set is (maybe?) isomorphic to a (probably finite) product of Maybe-outputs,
where each element of the product object looks like $(o_1~,~ .... ,~ o_k)$, where each spot in the
tuple corresponds to an input.

We conjecture that ledgers that have a partial product decomposition such that
the $\ups$ function can be expressed as a \emph{projection} are OD, so that

\[ \ups ~=~\pi_{TxS} \]

Note that this does not mean that this is simply a projection onto the second
coordinate. It rather means that all data needed to construct the
new state is stored in either the old state or transaction, and no computation outside
of projection is allowed.

Projections commute (except when $\Err$ occurs).

% inputs \in \Pi_{i~\in~\type{InputType}}, outputs
%
% inputs |-> outputs
%
% pi_{TxS} ((inputs_tx, outputs_tx), inputs |-> outputs) = \Pi_{i~\in~(inputs ~\setminus~ inputs_tx)} \pi
%

If we have transaction

\[ tx = (inputs~=~(txid, 1),~(txid, 2);~outputs~=~(txout_1,~txout_2)) \]

\begin{figure*}[htb]
  \emph{Projections}
  %
  \begin{equation*}
    \begin{array}{l@{~:~}l@{~~}ll}
    \pi_{TxS} & (\Tx,~\State) \to \State  \\
    & \text{projects data that will be in the state after applying $\ups$ }
    \end{array}
  \end{equation*}
  \emph{UTxO example}
  %
  \begin{equation*}
    \begin{array}{l@{~~}l@{~~}l@{~~}l@{~~}l@{~~}l@{~~}l@{~~}l@{~~}l}
      & txid_1 & txid_2 & txid_3 & ...\text{all possible $txid$s}... & txid_k   \\
      & v_1 & \Nt & v_3 & ...\text{values corresponding to above $txid$s}... & \Nt   \\
    \end{array}
  \end{equation*}
  \emph{View of UTxO by tx}
  %
  \begin{equation*}
    \begin{array}{l@{~~}l@{~~}l@{~~}l@{~~}l@{~~}l@{~~}l@{~~}l@{~~}l}
    & txid_1 & txid_2~(=~(txid, 1)) & txid_3~=(~(txin_1)) & ...\text{all possible $txid$s}... & txid_k~(=~(txid, 2))   \\
    & v_1 & txout_1 & v_3 & ...\text{values corresponding to above $txid$s}... & txout_2
    \end{array}
  \end{equation*}
  \caption{Transaction and state composition of OD ledgers}
  \label{fig:ud-comp}
\end{figure*}
