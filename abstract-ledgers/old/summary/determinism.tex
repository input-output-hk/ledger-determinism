\section{Deterministic Ledgers}
\label{sec:determinism}

\subsection{Stateful-determinism}
\label{sec:state-det}

We define a constraint in Figure \ref{fig:state-det}, and say that a ledger which
satisfies this constraint is \emph{statefully deterministic}.
The stateful determinism constraint requires that given any two valid ledger states $s,~s'$,
and a transaction $tx$ valid in both $s$ and $s'$, applying $tx$ to either state will result in the same
changes to the state. We compute using the function

\[ \Delta~\in~\seqof{(\type{Tx})} \to \type{ValSt} \to \Diff \]

Note that it is the changes made by a transaction that must be
the same -- the output states resulting from applying these changes may themselves
be distinct if $s \neq s'$.
We will later specify what exactly is the function that returns the ledger changes
made by a sequence of transactions, and what is the type $\Diff$ representing the changes.

\begin{figure*}[htb]
  \emph{Arbitrary state-determinism constraint on a ledger specification}
  %
  \begin{equation*}
    \begin{array}{l@{~~}l@{~~}ll}
    \fun{stateDetConstraint} ~=~ & \forall (s~s'~\in~\type{ValSt})~(tx~\in~\type{Tx}),~\\
    & ~~~~\ups~tx~s~\neq~\Err~\neq~\ups~tx~s \\
    & ~~~~\Rightarrow~~\Delta~[tx]~s~=~\Delta~[tx]~s'
    \end{array}
  \end{equation*}
  \caption{State-determinism}
  \label{fig:state-det}
\end{figure*}

\todopv{make a note here about only the transaction body making the changes,
which is a part of the tx that, if kept constant, makes the same s' from every s
regardless of the changes to the rest of the tx (and produces Err in the same
cases)}

What the state-determinism constraint says is that the update (or, change set)
to the state of a state-deterministic ledger is specified uniquely by the transaction list
$txs$ that is being applied, and is independent of the state to which it is applied in the
case that the update is valid in that state.

A definition of $\Delta$ making the constraint trivially true is easy to give.
However, in order for $\Delta$ to carry the intended meaning, the ledger
itself must admit a certain kind of structure, with which $\Delta$ indeed reduces to a
function that simply discards its second argument and returns its first.

In the general case of an arbitrary $L$, the definition of $\Delta$ may not have
this property. In fact, it should be very dependent on the specific data structures
of $\type{Tx}$ and $\type{State}$.

\subsection{Order-determinism}
\label{sec:order-det}

The differences between any two ledgers states $s,~ s'$ can be characterized fully
by a sequence of operations applied the trace of $s$ such that the output
sequence is a trace of the state $s'$. These operations are reordering, dropping,
and insertion of transactions.

The conception of determinism, which we call \emph{order-determinism} (OD), given Figure \ref{fig:order-det},
directly addresses the state differences caused by reordering only. That is,
if two valid states are generated by sequences of transactions that are permutations
of each other, OD constrains them to be the same state. Therefore, any additional transaction
applied to either state will also result in the same state, and the same changes, trivially
satisfying the state-determinism constraint

\[ (\fun{initState}~txs)~ =~ (\fun{initState}~txs') ~\Rightarrow ~\Delta~[tx]~(\fun{initState}~txs)~=~\Delta~[tx]~(\fun{initState}~txs') \]

\begin{figure*}[htb]
  \emph{Order-determinism constraint on a ledger specification}
  %
  \begin{equation*}
    \begin{array}{l@{~~}l@{~~}ll}
    \fun{orderDetConstraint} ~=~ & \forall~(s~\in~\type{ValSt})~(txs~\in~\seqof{\type{Tx}}),~\\
    & ~~~~txs'~\in~\fun{Permutation}~txs,~\\
    & ~~~~\fun{validUpdate}~s~txs~\neq~\Err~\neq~\fun{validUpdate}~s~txs' \\
    & ~~~~\Rightarrow~~\fun{validUpdate}~s~txs~=~\fun{validUpdate}~s~txs'
    \end{array}
  \end{equation*}
  \caption{Order-determinism}
  \label{fig:order-det}
\end{figure*}

Note here that, unlike in permutations groups, which are generated by all pairwise
permutations, it does not suffice for the OD constraint to require equality of
state for only pairwise swaps of transactions. This is because we only require
the output state to be the same for two transaction orderings if both produce
a valid state. This relation is not transitive. We can easily imagine a ledger
where the transaction sequences $[a, b, c, d]$  and $[c, a, d, b]$ produce
valid but not equal states, and all other permutations of $a, b, c, d$ produce
$\Err$. This ledger would still satisfy a pairwise constraint, but not
full OD.

Moreover, it does not make sense to drop the "both sequence orderings do not produce
an error" precondition, since real ledgers do, in fact, enforce partial orders
on transactions dependency (e.g., a ledger update would fail if a transaction spends
money from an account before another transaction first pays into it). It is also not
possible to ignore failing transactions in this definition of determinism since
dropping transaction from one of the sequences means that we now constraining
sequences containing different transactions (e.g. $a, b, c, d$ and $a, d$) to
producing the same output.

\todopv{can we somehow give similar characterizations that take into account inserting and dropping
transactions as well??}

\todopv{A key goal of this work is to give a local, as opposed to trace-based characterization
of determinism, so we want to connect state-determinism with order-determinism, and
find conditions under what ledgers are deterministic (without relying on analyzing
all traces, if possible)}

\subsection{Derivative structure in single-ledger categories}

A theory of changes and program derivatives
is presented in \cite{changes}. We use this theory to make precise the definition of $\Delta$,
and to describe the structure characteristic
of state-deterministic ledgers that admit a trivial $\Delta$.

This work gives a framework for defining \emph{derivatives}, where, for a given input,
a derivative maps changes to that input directly to changes in the output.
This approach adheres to some of the key ideas of the usual notion of differentiation
of functions, while adapting them to programs. Other abstract notions of differentiation,
such as presented in \cite{diffrestcats}, impose additional constraints on
the functions being differentiated, such as the definition of addition of functions.
The theory of changes outlined in \cite{changes} draws inspiration from an
abstract notion of derivation to deal with changes to data structures.

One noteworthy difference between the differentiation presented in the two papers is
the type of the derivative function. A differential category derivative of a
function $f : X \to Y$ has the type

\[ \type{D}[f] : X \times X \to Y \]

While the type of a derivative of a program $p : A \to B$ is

\[ \fun{diff}~p : A \to \Diff~A \to \Diff~B \]

The reason for this is that to define a coherent and applicable theory of changes
to a data structure, we want to have a special type $\Diff~A$ for each
structure $A$ that can express changes to $A$. Additional structure required
for functional differentiation allows for a theory wherein the changes to a term of a type
can be expressed by another term of that same type.

Now that we gave some justification for discussing the type of changes to a ledger
state, we can explore what that looks like. The change type operator, $\Diff$, introduced
in \cite{changes}, along with its behaviour and the constraints on it, is specified in Figure \ref{fig:diff}.
Here we tailor the definition to a single ledger specification, so that it only
describes the change type of the ledger state type for a given ledger specification $L$.

In the work mentioned above, all functions are total, as they are applied only
to the domains (sets) in which they are valid.
However, in our model, the partiality of transaction application
to the ledger state object is integral to the discussion of determinism.
Disallowing the application of certain updates to certain states is what
allow us to guarantee deterministic updates in the valid update cases --- so, we
must be able to reason about the error cases.

Moreover, we want to be able to discuss
what it means for a particular category of valid ledger states and maps between them
to enjoy differential structure even if applying certain changes to a given
state results in a failed ($\Err$) update.

For this reason, we additionally make the change that only some changes are permitted, i.e. the
ones that lead to a valid ledger state, so that $\fun{applyDiff}$ returns
a error-type.

\begin{figure*}[htb]
  \emph{$\Diff$ Type Accessors}
  %
  \begin{equation*}
    \begin{array}{r@{~\in~}l@{~~~~}lr}
      \Diff
      & \Type
      & \text{The change type of the ledger state}
    \end{array}
  \end{equation*}
  \emph{$\Diff$ functions}
  %
  \begin{equation*}
    \begin{array}{r@{~\in~}l@{~~~~}lr}
      \fun{applyDiff} & \type{State} \to \Diff \to \type{State} &
      \text{Apply a set of changes to a given state}
      \nextdef
      \fun{extend} & \Diff \to \Diff \to \Diff &
      \text{Compose sets of changes}
      \nextdef
      \fun{zero} & \Diff  &
      \text{No-changes term}
    \end{array}
  \end{equation*}
  \emph{$\Diff$ constraints}
  %
  \begin{align*}
      & \fun{zeroChanges} ~\in~\forall~s, \fun{applyDiff}~s~\fun{zero}~=~s  \\
      & \text{Applying the zero change set results in no changes}
      \nextdef
      &\fun{applyExtend} ~\in~ \forall~s~txs1~txs2,~\fun{validUpdate}~s~txs1~\neq~\Err~\neq~\fun{validUpdate}~s~txs2, \\
      &~~~~\fun{applyDiff}~s~ (\fun{extend}~txs2~txs1)~=~\fun{applyDiff} (\fun{applyDiff}~s~txs1)~txs2 \\
      & \text{If both change sets are valid, composing them gives the same result as applying them in sequence}
  \end{align*}
  \caption{Specification for a change type $D \in~\Diff$}
  \label{fig:diff}
\end{figure*}

Now, we want to introduce the idea of taking and evaluating derivatives in
a single-ledger category for a given ledger $L~\in~\LS$, see Figure \ref{fig:diff-cat}.

\todopv{derivativeConstraint is wrong! we want to say something like : for any interleaving
of txs and ds changes, the equality must hold - how can we express this categorically?}

\begin{figure*}[htb]
  \emph{$\DCat$ Type Accessors}
  %
  \begin{equation*}
    \begin{array}{r@{~\in~}l@{~~~~}lr}
      \type{DerType}
      & \Type
      & \text{The type representing derivative functions}
    \end{array}
  \end{equation*}
  \emph{$\DCat$ functions}
  %
  \begin{equation*}
    \begin{array}{r@{~\in~}l@{~~~~}lr}
      \fun{takeDer} & \seqof{(\type{Tx})} \to \type{DerType} &
      \text{Take the derivative of a transaction}
      \nextdef
      \fun{evalDer} & \type{DerType} \to \type{State} \to \Diff \to \Diff &
      \text{Calculate the diff value of a derivative at $(s,~ds)$}
    \end{array}
  \end{equation*}
  \emph{$\DCat$ constraints}
  %
  \begin{align*}
      &\fun{derivativeConstraint} ~\in~ \forall~s~ds~txs,\\
      &~~~~\fun{validUpdate} ~(\fun{applyDiff}~ ds~ s) ~txs~\neq~\Err, \\
      &~~~~\fun{applyDiff} ~(\fun{evalDer} ~(\fun{takeDer}~ txs) ~s~ ds)~ (\fun{validUpdate}~ s ~txs)~\neq~\Err, \\
      &~~~~\fun{validUpdate} ~(\fun{applyDiff}~ ds~ s) ~txs ~= \\
      &~~~~~~~~ \fun{applyDiff} ~(\fun{evalDer} ~(\fun{takeDer}~ txs) ~s~ ds)~ (\fun{validUpdate}~ s ~txs) \\
      & \text{If both change sets are valid, composing them gives the same result as applying them in sequence}
  \end{align*}
  \caption{Structure $DC~\in~\DCat$ for a data-differentiable category }
  \label{fig:diff-cat}
\end{figure*}


\subsection{Instantiating derivative structure}

\begin{figure*}[htb]
  \emph{Data-differentiable structure}
  \begin{equation*}
    \begin{array}{l@{~\leteq~}l@{~~~~}lr}
      \Diff~&~\seqof{(\type{Tx})} \\
      \fun{applyDiff}~&~\fun{flip}~\fun{validUpdate} \\
      \fun{extend} ~&~\fun{flip}~(++) \\
      \fun{zero}~&~[]
    \end{array}
  \end{equation*}
  \emph{$\DCat$ structure instantiation $\DCat$}
  \begin{equation*}
    \begin{array}{l@{~\leteq~}l@{~~~~}lr}
      \type{DerType}~&~\Nt \\
      \fun{takeDer}~txs~&~\Nt \\
      \fun{evalDer}~txs~s~ds~&~\begin{cases}
        ds & \text{ if } \fun{validUpdate}~s~ds~\neq~\Err \\
        \Err & \text{ otherwise}
      \end{cases} \\
    \end{array}
  \end{equation*}
  \caption{Instantiation of data-differentiable structure in $\type{SLC}$ }
  \label{fig:diff-cat-inst}
\end{figure*}

\todopv{evalDer definition might also be wrong here w.r.t. Err}

We give one way to instantiate the data-differentiable structure in a single-ledger
category $\type{SLC}$ in Figure \ref{fig:diff-cat-inst}, which we will denote
$\DCat$.

This instantiation of the $\Diff$ structure is common to all $\type{SLC}$ categories,
and trivially satisfies the constraints in Figure \ref{fig:diff}.
The type representing
changes to the (valid or $\Err$) ledger state, $\Diff$, is $\seqof{(\type{Tx})}$
because lists of transactions is exactly the data structure that represents
changes to the state. Applying changes is therefore represented by transaction application
$\fun{validUpdate}$. Extending one change set with another is concatenation of
the two transaction lists, and the empty change set is the nil list.

The $\DCat$ structure we specified, however, is not necessarily
commonly admissible as a $\DCat$ instantiation in all ledgers,
as it does not guarantee that the $\fun{derivativeConstraint}$ will always be satisfied.
We will now discuss the relation of the specification we give, the satisfiability of
the derivative constraint, and both versions of the determinism definition.

\subsection{$\Delta$ and derivation structure specification in ledgers}

In this section we present and justify our choice of $\Delta$ definition, as well
as use of $\DCat$ structure.

The idea of derivation, as presented in \cite{changes}, is that,
given a function $f$ and an input $s$ to that function, takes a change set $ds$ to another change set $ds'$
such that the changes $ds'$ are consistent with the changes of the original $ds$,
but done after $f$ has been applied --- to the new state output by $f~s$.

Recall that, intuitively, the definition of \emph{state-determinism} conveys that the changes
a transaction makes \emph{are specified entirely within the transaction itself}.

\todopv{this needs more explanation about the connection between Delta and the
derivative constraint and evalDer being a proj function AND figure out what
to do about $\Err$}

The derivative constraint then reduces to, for all $s,~txs,~ds$,

\[\text{(a)~:~~~} \fun{validUpdate}~ (\fun{validUpdate}~ ds~ s) ~txs~=~\fun{validUpdate} ~ds~(\fun{validUpdate}~ s ~txs) \]

whenever $\Err$ is not produced by any computation. In particular,

\[ \forall~s,~tx,~\fun{evalDer} ~(\fun{takeDer}~ []) ~s~ [tx] ~=~[tx] \]


\subsection{Order-determinism, state-determinism, and derivation.~}
\label{sec:od-sd-d}


\todopv{The proof in this section does not work without a good definition of
differentiation for ledgers with $\Err$ outputs. We want to be able to
prove that if a ledger $L$ is order-determinismic, it admits data-differentiable
category structure with constant derivaties, and maybe even IFF}

\todopv{In this section, we want to specify for what kinds of ledgers do change sets correspond exactly
to the lists of transactions (i.e. the derivatives should be constant)? transactions i think always
map onto change sets, but we want the change set to be represented exactly by the transaction,
independent of the state they are applied to}

\begin{theorem}
  Suppose $L$ is a ledger, and $\DCat$ structure is as specified in Figure \ref{fig:dcat-pf}.
  Then,

  \[ \fun{orderDetConstraint} ~\Rightarrow~\type{SLC}~\in~\DCat \]

  \label{theo:dcat-pf}
\end{theorem}

\begin{proof}
  For an arbitrary $s,~txs,~ds$, we can instantiate the variables in the
  order-determinism definition in Figure \ref{fig:order-det} by

  \[ \text{(b)~:~~~}\fun{orderDetConstraint}~s~(\fun{exted}~txs~ds)~(\fun{extend}~ds~txs) \]

  Since

  \[ \fun{extend}~ds~txs~\in~\fun{Permutation}~(\fun{extend}~txs~ds) \]

  Whenever (b) holds for $s,~txs,~ds$, it immediately follows that so does $\fun{derivativeConstraint}$ .
\end{proof}

\subsection{Examples.~}
\label{sec:det-examples}

\todopv{Example (iv) should work if we had a proper definition of differentiation for ledgers
that output $\Err$ - we want a definition that lets us reason about such cases.}

In some of the following examples, we use an orthogonal but complementary context
to determinism --- the notion of \emph{replay protection}.

\begin{definition}
  In a ledger with \emph{replay protection},

  \[ \forall~lstx,~tx,~tx~\in~lstx ~ \Rightarrow~\fun{validUpdate}~\fun{initState}~(lstx~++~[tx])~=~\Err \]
\end{definition}

This is a valuable property to have to avoid adversarial or accidental replaying of transactions.

\begin{itemize}
  \item[(ii)] \emph{Account ledger with value proofs.} A ledger $AP$ where the state consists of a
  map $\State~\leteq~\type{AccID}~\mapsto~\type{Assets}$, and $\Tx~\leteq~(\State,~\State)$.
  If for a given state and transaction $s,~tx$, $\fun{fst}~tx~\subseteq~s$, the
  update function outputs $(s~\setminus~\fun{fst}~tx)~\cup~(\fun{snd}~tx)$. Presumably,
  there will also be a preservation of value condition imposed on the asset total.

  This ledger provides no replay protection --- one can easily submit transactions
  that keep adding and removing the same key-value pairs.

  Note here that a regular account-based ledger is also deterministic, but not when
  smart contracts or case analysis is added (eg. the $\ups$ allows for transfer
  for a quantity $q$ \emph{or} any amount under $q$ if $q$ is not available). This
  value-proof feature makes it possible to enforce determinism even in such
  conditions of multiple control flow branches. Account-based ledgers also, unlike
  the UTxO model, do not offer a simple, space-efficient solution to replay protection.
  Ethereum does offer it, but it is somewhat convoluted, see EIP-155. \\

  \item[(iii)] \emph{Account ledger with value proofs and replay protection}. It is easy to add
  replay protection to the previous example. For example, by extending the data
  stored in the state,

  \[ \State~\leteq~([\Tx],~\type{AccID}~\mapsto~\type{Assets}) \]

  The update function will then have an additional check that a given $tx~\notin~\fun{txList}~s$. \\

  \item[(iv)] \emph{Non-deterministic differentiable ledger.} It is possible to have a ledger
  that is differentiable, has replay protection, but is not deterministic. Let us
  extend the above state with a boolean,

  \[ \State~\leteq~(\Bool,~[\Tx],~\type{AccID}~\mapsto~\type{Assets}) \]

  \[ \Tx~\leteq~((\Bool,~\Bool),~(\State,~\State)) \]

  And specify $\ups~s~=~s'$ in a way that the boolean of the state $s$ is updated in $s'$ whenever
  the first boolean in the transaction matches the one in the state :

  \[ \fun{bool}~s' ~\leteq~ \begin{cases}
     \fun{snd}~(\fun{bool}~tx) & \text{ if } \fun{fst}~(\fun{bool}~tx)~==~\fun{bool}~s \\
     \fun{bool}~s & \text{ otherwise}
  \end{cases} \]

  We can define differentiation for such a ledger in a way that allows switching the order of application
  of functions $ds$ and $txs$ by using re-interpreting the changes $txs$ makes when $ds$ is applied
  first so that the $\fun{derivativeConstraint}$ is satisfied :

    \[\fun{evalDer} ~txs ~s~ ds ~\leteq~ ds++ [txFlip] \]

    where

    \[ txFlip~=~((\fun{bool}~(s~txs),~\fun{bool}~(s~ds~txs)),~\Nt) \]

    The $\fun{evalDer}$ function cannot be defined to be a projection of the third
    coordinate, as this does not satisfy the constraint. It is straightforward to change the $\ups$
    function of this ledger
    such that a valid derivative function can be a simple projection, and the ledger - deterministic. The $\ups$ function must
    produce an $\Err$ instead of
    allowing the state update to take place even if $\fun{fst}~(\fun{bool}~tx)~==~\fun{bool}~s$ does not hold.


\end{itemize}
