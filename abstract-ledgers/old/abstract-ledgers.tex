\documentclass[runningheads]{llncs}

% correct bad hyphenation here
\hyphenation{}

%\usepackage{natbib}
\usepackage{url}

% *** MATHS PACKAGES ***
%
\usepackage[cmex10]{amsmath}
\usepackage{amssymb}
\usepackage{stmaryrd}
%\usepackage{amsthm}

\usepackage[margin=2.5cm]{geometry}
\usepackage{iohk}
\usepackage{microtype}
\usepackage{mathpazo} % nice fonts
\usepackage{amsmath}
\usepackage{amssymb}
%\usepackage{amsthm}
\usepackage{latexsym}
\usepackage{mathtools}
\usepackage{stmaryrd}
\usepackage{extarrows}
\usepackage{slashed}
\usepackage[unicode=true,pdftex,pdfa,colorlinks=true]{hyperref}
\usepackage{xcolor}
\usepackage[capitalise,noabbrev,nameinlink]{cleveref}
\usepackage{float}
\usepackage{upgreek,textgreek}

% *** ALIGNMENT PACKAGES ***
%
\usepackage{array}
\usepackage{float}  %% Try to improve placement of figures.  Doesn't work well with subcaption package.
\usepackage{subcaption}
\usepackage{caption}

\usepackage{subfiles}
\usepackage{geometry}
\usepackage{listings}
% \usepackage[dvipsnames]{xcolor}
\usepackage{verbatim}
\usepackage{listings}% http://ctan.org/pkg/listings
\lstset{
  basicstyle=\ttfamily,
  mathescape
}
\usepackage{alltt}
\usepackage{paralist}

\usepackage{todonotes}
%\usepackage[disable]{todonotes}

% This has to go at the end of the packages.
%\usepackage[colorlinks=true,linkcolor=MidnightBlue,citecolor=ForestGreen,urlcolor=Plum]{hyperref}

% Stuff for splitting figures over page breaks
%\DeclareCaptionLabelFormat{continued}{#1~#2 (Continued)}
%\captionsetup[ContinuedFloat]{labelformat=continued}

% *** MACROS ***

%\newcommand\coqRepo{https://github.com/input-output-hk/state-machine-research/tree/master/Abstract%20Ledger%20Specification}

\newcommand{\todochak}[1]{\todo[inline,color=purple!40,author=chak]{#1}}
\newcommand{\todompj}[1]{\todo[inline,color=yellow!40,author=Michael]{#1}}
\newcommand{\todokwxm}[1]{\todo[inline,color=blue!20,author=kwxm]{#1}}
\newcommand{\todopv}[1]{\todo[inline,color=purple!40,author=Polina]{#1}}

\newcommand{\red}[1]{\textcolor{red}{#1}}
\newcommand{\redfootnote}[1]{\red{\footnote{\red{#1}}}}
\newcommand{\blue}[1]{\textcolor{blue}{#1}}
\newcommand{\bluefootnote}[1]{\blue{\footnote{\blue{#1}}}}

%% A version of ^{\prime} for use in text mode
\makeatletter
\DeclareTextCommand{\textprime}{\encodingdefault}{%
  \mbox{$\m@th'\kern-\scriptspace$}%
}
\makeatother

\newcommand{\code}{\texttt}
\renewcommand{\i}{\textit}  % Just to speed up typing: replace these in the final version
\renewcommand{\t}{\texttt}  % Just to speed up typing: replace these in the final version
\newcommand{\s}{\textsf}  % Just to speed up typing: replace these in the final version
\newcommand{\msf}[1]{\ensuremath{\mathsf{#1}}}
\newcommand{\mi}[1]{\ensuremath{\mathit{#1}}}

%% A figure with rules above and below.
\newcommand\rfskip{3pt}
%\newenvironment{ruledfigure}[1]{\begin{figure}[#1]\hrule\vspace{\rfskip}}{\vspace{\rfskip}\hrule\end{figure}}
\newenvironment{ruledfigure}[1]{\begin{figure}[#1]}{\end{figure}}

%% Various text macros
\newcommand{\true}{\textsf{true}}
\newcommand{\false}{\textsf{false}}

\newcommand{\hash}[1]{\ensuremath{#1^{\#}}}

\newcommand\mapsTo{\ensuremath{\mapsto}}
\newcommand\cL{\ensuremath{\{}}
\newcommand\cR{\ensuremath{\}}}

\newcommand{\List}[1]{\ensuremath{\s{List}[#1]}}
\newcommand{\Set}[1]{\ensuremath{\s{Set}[#1]}}
\newcommand{\FinSet}[1]{\ensuremath{\s{FinSet}[#1]}}
\newcommand{\Interval}[1]{\ensuremath{\s{Interval}[#1]}}
\newcommand{\FinSup}[2]{\ensuremath{\s{FinSup}[#1,\linebreak[0]#2]}}
% ^ \linebeak is to avoid a bad line break when we talk about finite
% maps.  We may be able to remove it in the final version.
\newcommand{\supp}{\msf{supp}}

\newcommand{\FPScript}{\ensuremath{\s{Script}}}
\newcommand{\Script}{\FPScript}
\newcommand{\scriptAddr}{\msf{scriptAddr}}
\newcommand{\ctx}{\ensuremath{\s{Context}}}
\newcommand{\toData}{\ensuremath{\s{toData}}}
\newcommand{\fromData}{\msf{fromData}}

\newcommand{\verify}{\msf{verify}}

\newcommand{\mkContext}{\ensuremath{\s{mkContext}}}

\newcommand{\applyScript}[1]{\ensuremath{\llbracket#1\rrbracket}}

% Macros for eutxo things.
\newcommand{\tx}{\mi{tx}}
\newcommand{\TxId}{\ensuremath{\s{TxId}}}
\newcommand{\txId}{\msf{txId}}
\newcommand{\txrefid}{\mi{id}}
\newcommand{\Address}{\ensuremath{\s{Address}}}
\newcommand{\DataHash}{\ensuremath{\s{DataHash}}}
\newcommand{\hashData}{\msf{dataHash}}
\newcommand{\idx}{\mi{index}}
\newcommand{\inputs}{\mi{inputs}}
\newcommand{\outputs}{\mi{outputs}}
\newcommand{\validityInterval}{\mi{validityInterval}}
\newcommand{\scripts}{\mi{scripts}}
\newcommand{\forge}{\mi{forge}}
\newcommand{\sigs}{\mi{sigs}}
\newcommand{\fee}{\mi{fee}}
\newcommand{\addr}{\mi{addr}}
\newcommand{\pubkey}{\mi{pubkey}}
\newcommand{\val}{\mi{value}}  %% \value is already defined
\newcommand{\Value}{\type{Value}}  %% \value is already defined

\newcommand{\validator}{\mi{validator}}
\newcommand{\redeemer}{\mi{redeemer}}
\newcommand{\datum}{\mi{datum}}
\newcommand{\datumHash}{\mi{datumHash}}
\newcommand{\datumWits}{\mi{datumWitnesses}}
\newcommand{\Data}{\ensuremath{\s{Data}}}
\newcommand{\Input}{\ensuremath{\s{Input}}}
\newcommand{\Output}{\ensuremath{\s{Output}}}
\newcommand{\OutputRef}{\ensuremath{\s{OutputRef}}}
\newcommand{\Signature}{\ensuremath{\s{Signature}}}
\newcommand{\Ledger}{\ensuremath{\s{Ledger}}}

\newcommand{\outputref}{\mi{outputRef}}
\newcommand{\txin}{\mi{in}}
\newcommand{\id}{\mi{id}}
\newcommand{\lookupTx}{\msf{lookupTx}}
\newcommand{\getSpent}{\msf{getSpentOutput}}

\newcommand{\consumes}[1]{\msf{consumes(#1)}}
\newcommand{\consumesOne}[1]{\msf{consumesOne(#1)}}
\newcommand{\cid}{\mi{cid}}
\newcommand{\inputValue}{\mi{inputValue}}
\newcommand{\rMin}{r_{\mi{min}}}
\newcommand{\rMax}{r_{\mi{max}}}

\newcommand{\Tick}{\ensuremath{\s{Tick}}}
\newcommand{\currentTick}{\msf{currentTick}}
\newcommand{\spent}{\msf{spentOutputs}}
\newcommand{\unspent}{\msf{unspentOutputs}}
\newcommand{\txunspent}{\msf{unspentTxOutputs}}
\newcommand{\utxotx}{\msf{Tx}}

\newcommand{\Quantity}{\ensuremath{\s{Quantity}}}
\newcommand{\Asset}{\ensuremath{\s{Asset}}}
\newcommand{\Policy}{\ensuremath{\s{PolicyID}}}
\newcommand{\Quantities}{\ensuremath{\s{Quantities}}}
\newcommand{\nativeCur}{\ensuremath{\mathrm{nativeC}}}
\newcommand{\nativeTok}{\ensuremath{\mathrm{nativeT}}}

\newcommand{\PublicKey}{\ensuremath{\s{PubKey}}}
\newcommand{\PrivateKey}{\ensuremath{\s{PrivateKey}}}

\newcommand{\pkey}{\ensuremath{\pi_{\mathsf{p}}}}
\newcommand{\skey}{\ensuremath{\pi_{\mathsf{s}}}}

\newcommand\B{\ensuremath{\mathbb{B}}}
\newcommand\N{\ensuremath{\mathbb{N}}}
\newcommand\Z{\ensuremath{\mathbb{Z}}}
\renewcommand\H{\ensuremath{\mathbb{H}}}
%% \H is usually the Hungarian double acute accent
\newcommand{\emptyBs}{\ensuremath{\emptyset}}
\newcommand{\leteq}{\ensuremath{\mathrel{\mathop:}=}}
\newcommand{\Nt}{\ensuremath{\Diamond}}
\newcommand{\Bool}{\type{Bool}}
\newcommand{\Type}{\type{Type}}
\newcommand{\Diff}{\type{Diff}}
\newcommand{\ValSt}{\type{ValSt}}
\newcommand{\Block}{\type{Block}}

%ledger spec commands
\newcommand{\LS}{\mathcal{L}\mathcal{S}}
\newcommand{\State}{\type{State}}
\newcommand{\Tx}{\type{Tx}}
\newcommand{\ID}{\type{ID}}
\newcommand{\Th}{\mathcal{T}}
\newcommand{\Tu}{\mathcal{U}}
\newcommand{\T}{\type{T}}
\newcommand{\Err}{\type{Err}}
\newcommand{\DCat}{\type{DCat}}
\newcommand{\ups}{\fun{update}}
\newcommand{\Time}{\type{Time}}
\newcommand{\True}{\type{True}}
\newcommand{\False}{\type{False}}


\usepackage{etoolbox}

% For anonymisation
\newtoggle{anonymous}
\toggletrue{anonymous}
\iftoggle{anonymous}{
  \newcommand{\Cardano}{CHAIN}
  \newcommand{\Plutus}{LANG}
}{
  \newcommand{\Cardano}{Cardano}
  \newcommand{\Plutus}{Plutus Core}
}

% Names, for consistency
\newcommand{\UTxO}{\type{UTxO}}
\newcommand{\Slot}{\type{Slot}}
\newcommand{\Block}{\type{Block}}
\newcommand{\UTXOma}{UTXO$_{\textsf{ma}}$}
\newcommand{\EUTXO}{E\UTXO{}}
\newcommand{\ExUTXO}{Extended \UTXO{}}
\newcommand{\CEM}{CEM}

% relaxed float placement
\renewcommand{\topfraction}{.95}
\renewcommand{\bottomfraction}{.7}
\renewcommand{\textfraction}{.15}
\renewcommand{\floatpagefraction}{.66}
\renewcommand{\dbltopfraction}{.66}
\renewcommand{\dblfloatpagefraction}{.66}
\setcounter{topnumber}{9}
\setcounter{bottomnumber}{9}
\setcounter{totalnumber}{20}
\setcounter{dbltopnumber}{9}

%\newtheorem{theorem}{Theorem}[theorem]

%% ------------- Start of document ------------- %%

\begin{document}

\lstset{
  basicstyle=\ttfamily,
  columns=fullflexible,
  keepspaces=true,
}

\title{Abstract Ledger Specifications}
%\title{\UTXOma: \UTXO\ with Multi-Asset Support\\ --- DRAFT --- DRAFT ---}

% First names are abbreviated in the running head.
% If there are more than two authors, 'et al.' is used.

\author{
  Polina Vinogradova\inst{1}
  \and
  IOG folk
  \and
  Univerity of Edinburgh folk
}

\authorrunning{IOG and UofE folk}
%\authorrunning{--- DRAFT --- DRAFT --- DRAFT ---}

\institute{
  IOG,
  \email{firstname.lastname@iohk.io}
  \and
  University of Edinburgh,
  \email{orestis.melkonian@ed.ac.uk, wadler@inf.ed.ac.uk}
}

\maketitle


\begin{abstract}

Ensuring deterministic behaviour in distributed blockchain ledger design matters to end users because
it allows for locally predictable fees and smart contract evaluation outcomes.
In this work we begin by establishing a formal definition of an abstract ledger together with
a ledger update procedure. We use this model as a basis for formalizing and studying
several definitions of the collection of properties colloquially classified as
determinism. We investigate the relations between these properties, and the implications
each of these has for design and static analysis of ledgers, giving simple but
illustrative examples of ledgers with and without these properties.
We discuss how these ideas can be applied to realistic, deployable ledger
design and architecture.



\end{abstract}

\keywords{blockchain \and formal methods \and ledger}

\section{Introduction}
\label{sec:intro}

In the context of blockchain transaction processing and smart contract execution,
determinism is usually taken to mean something like "the ability to predict locally,
before submitting a transaction, the on-chain result of processing that transaction and its scripts".
This is an important aspect of ledger design because users care about being able to accurately predict
fees they will be charged, rewards they will receive from staking, outcomes of
smart contract executions, etc. before submitting transactions. The purpose of this
work is to formalize this property of ledgers, and study the constraints under
which it can be guaranteed, thereby providing analysis tools and design principles for building ledgers
whose transaction processing outcomes can be accurately forecast.

Blockchain ledger and consensus design relies on the definition of the transaction processing function
itself being entirely deterministic. The impossibility of predicting
the exact on-chain state transactions will be applied to, however, is due to
unpredictable network propagation of transactions, resulting in an arbitrary
order in which they are processed as the source of non-determinism. Determinism
can, therefore, be formulated entirely in the language of transaction application commutativity,
but we retain the commonly used blockchain terminology in this work.

We present an abstract
ledger model capturing the architectural core shared by most blockchain platforms:
a ledger is a state transition system, with valid transactions (or blocks) as the only transitions.
We then formalize the definition of determinism in terms of this abstract functional
specification of ledger structure, and use mathematical tools
for analyzing them in order is to establish a way to reason about
conditions under which the transaction order has no effect on the resulting state
or parts thereof.

A similar construction to our ledger definition is presented in \cite{paxos}.
However, the focus there is on the ordering of state updates
to optimize the execution of a particular consensus algorithm.
Another related study of commutativity of state transitions is ~\cite{commautomata}.

\section{The Abstract Ledger Model}
\label{sec:model}


Each instance of $\LS$ (ledger specification, or ledger) requires the structure
given in Figure \ref{fig:ledger-spec} to be specified.

We use the $A_{\Err}~=~A~\uniondistinct~\Err$ to represent a non-$A$ error term in
addition to terms of the $A$ type.

We omit the subscript $_L$ whenever $L$ is implied, eg. $\State$ for $\State_L$.
We also use the shorthand here

\[ (s~ds)~:=~\fun{validUpdate}_L~ s~ ds \]

\todopv{Need better notation than this one (above)}

Given $L$, we denote the set of dependent pairs, together with the $\Err$ state, by $\ValSt_{\Err}$, where

\begin{itemize}
  \item the first term of the dependent pair is a state $s : \State_L$, and
  \item the second is a proof that there exists a list of transactions that, applied in sequence to
 the initial state, produce the state $s$. We call such a list the \emph{trace} of $s$.
\end{itemize}

We call $s : \State_L$ a \emph{valid state} whenever $(s, pf)~:~\ValSt$ for some
$pf$ specifying a trace of $s$. The state $s$ may have more than one trace.

\begin{figure*}[htb]
  \emph{$\LS$ Type Accessors}
  %
  \begin{equation*}
    \begin{array}{r@{~\in~}l@{~~~~}lr}
      \type{Tx}_L
      & \type{Set}
      & \text{The set of transactions} \\
      \type{State}_L
      & \type{Set}
      & \text{The set of ledger states} \\
    \end{array}
  \end{equation*}
  \emph{$\LS$ functions}
  %
  \begin{equation*}
    \begin{array}{r@{~\in~}l@{~~~~}lr}
    \ups_L & ~\type{Tx}_L \to \type{State}_L \to (\type{State}_L)_{\Err}
    & \text{Updates state with a given transaction} \\
    \fun{initState}_L & ~\type{State}_L
    & \text{Gives the initial state of the specification} \\
    \end{array}
  \end{equation*}
  \caption{Ledger Specification for a ledger $L\in~\LS$}
  \label{fig:ledger-spec}
\end{figure*}

\begin{figure*}[htb]
  \emph{State Computation}
  %
  \begin{align*}
    & \fun{updateErr}_L ~\in~ \type{Tx}_L \to (\type{State}_L)_{\Err} \to (\type{State}_L)_{\Err} \\
    & \fun{updateErr}_L ~tx~s~ ~=~
        \begin{cases}
          \Err & \text{if~} s = \Err \\
          \ups~tx~s & \text{otherwise}
        \end{cases} \\
    & \text{Updates a error state} \\~\\
    %
    & \fun{computeState}_L ~\in~ \type{State}_L \to \seqof{(\type{Tx}_L)} \to (\type{State}_L)_{\Err} \\
    & \fun{computeState}_L~ initS ~txs ~=~
        \fun{foldl}~\fun{updateErr}_L~initS~txs \\
    & \text{Applies a list of transactions to the initial state} \\~\\
    %
    & \type{ValSt}_L ~\in~ \type{Set} \\
    & \type{ValSt}_L ~=~
        \{ ~s~ \mid ~\fun{validState}_L~s~\}_{\Err} \\
    & \text{Set of all valid states of $L$} \\~\\
    %
    & \fun{validUpdate}_L ~\in~ \type{ValSt}_L \to \seqof{(\type{Tx}_L)} \to \type{ValSt}_L\\
    & \fun{validUpdate}_L~s~txs ~=~\\
        &~~~~\begin{cases}
          \Err & \text{if~} s = \Err \\
          (\fun{computeState}_L~s~txs,~\fun{proveVl}_L~s~txs) & \text{otherwise}
        \end{cases} \\
    & \text{A valid state is updated to another valid state or $\Err$} \\~\\
  \end{align*}
  \emph{Validity of ledgers, transactions, and states}
  %
  \begin{align*}
    & \fun{validTx}_L ~\in~ \type{Tx}_L \to \type{State}_L \to \Bool \\
    & \fun{validTx}_L~tx~s~ = ~\fun{update}_L ~tx~s~ \neq ~\Err \\
    & \text{Transaction updates a state non-trivially} \\~\\
    %
    & \fun{valid}_L ~\in~ \seqof{(\type{Tx}_L)} \to \Bool \\
    & \fun{valid}_L~txs~ = ~\fun{computeState}_L ~txs~ \neq ~\Err \\
    & \text{Computed state is non-trivial} \\~\\
    %
    & \fun{validState}_L ~\in~ \type{State}_L \to \type{Prop} \\
    & \fun{validState}_L~s~ = ~\exists~~txs~\in~\seqof{(\type{Tx}_L)},~
      s~=~\fun{computeState}_L~(\fun{initState}_L) ~txs \\
    & \text{Given state is non-trivial}
  \end{align*}
  \caption{$\LS$ auxiliary functions and definitions}
  \label{fig:ledger-aux}
\end{figure*}

Proof-constructing function

\[\fun{proveVl}_L ~\in~ \type{ValSt}_L \to \seqof{(\type{Tx}_L)} \to \]
\[ (\fun{validState}_L~s) \vee (\fun{computeState}_L~s~txs~=~\Err)\]

\todopv{Proof of $\fun{proveVl}_L~s~txs$ goes here. See Coq formalization.}


\textbf{Errors in ledger updates.}

\todopv{Important, cant compare reorderings without them. }

\section{Categories of Ledgers}
\label{sec:cats}

There are different categories one may construct out of a ledger specification.

\textbf{Proposition : Valid states as pullbacks in $\Set$. } Suppose
$\mathcal{P}$ is a pullback in $\Set$, \newline

\begin{center}
  \xymatrix{
      \mathcal{P} \pullbackcorner \ar[r]^{\pi_2} \ar[d]_{\pi_1}   & [\Tx] \ar[d]^{\fun{initState} \odot \wcard} \\
      \State \ar[r]^{\fun{id}_{\State}}                         & \State
  }
\end{center}

or a given ledger $L = (\State, \Tx, \odot, \fun{initState})$. Then the set of all
valid states $\V = \pi_2 \mathcal{P}$.

\textbf{Proof. } Recall that

\[ \V~=~ \{ ~s~ \mid ~\exists~\var{lstx}~\in~[Tx],~\fun{initState}~\odot~\var{lstx}~=~s~\} \]

Recall that a pullback in $\Set$ is defined by

\[ \mathcal{P}~=~ \{ ~(txs, s)~ \mid ~\fun{initState}~\odot~\var{txs}~=~s~\} \]

Now, for any $s \in \V$, we pick a $txs$ such that $\fun{initState}~\odot~\var{txs}~=~s$,
since such a $txs$ exists. So, $(txs, s) \in \mathcal{P} ~~\Rightarrow ~~s \in \pi_2 \mathcal{P}$.

For any $s \in \pi_2 \mathcal{P}$, there exists $(txs, s) \in \mathcal{P}$ such that
$\pi_2 (txs, s) = s$, and $\fun{initState}~\odot~\var{txs}~=~s$, which proves the result.

\subsection{Single-ledger category}
\label{sec:slc}

Given a ledger $L = (\State, \Tx, \odot, \fun{initState})$, we define the category $\type{SL}$, constructed
from a single ledger specification's types, transitions, and initial state,
in the following way :

\begin{itemize}
  \item[(i)] Objects : $\fun{Obj}~\type{SL} = \State$ \newline

  \item[(ii)] Morphisms : $\fun{Hom}~S1~S2 = \{ ~\wcard~\odot~txs \mid ~txs \in [\Tx]~,~S1~ \odot~txs~=~S2 \}$ \newline

  \item[(iii)] Identity : $\fun{id}_{S1} = [] ~\in~\fun{Hom}~S1~S1 $ \newline

  \item[(iv)] Composition : $\forall~txs1,~txs2, S$ \\
  $(\wcard~\odot~txs2) ~\circ~(\wcard~\odot~txs1) (S) = (\wcard~\odot~txs2) (S~\odot~txs1) = (S~\odot~txs1) \odot txs2  \\
  ~~~~ =  S~\odot~(txs1 ++ txs2)~~\Rightarrow ~~(\wcard~\odot~(txs1 ++ txs2))~\in ~\fun{Hom}~S1~S3$

  \item[(v)] Associativity : holds due to the $\fun{ledgerAssoc}$ constraint.

  % $\forall~txs1,~txs2,~tx3, S$ \\
  % $(\wcard~\odot~txs3) ~\circ~((\wcard~\odot~txs2) ~\circ~(\wcard~\odot~txs1)) (S)
  % = (\wcard~\odot~txs3) ((\wcard~\odot~txs2) (\wcard~\odot~txs1) (S)) \\
  % = (\wcard~\odot~txs3) ((\wcard~\odot~txs2) (S~\odot~txs1)) =  (\wcard~\odot~txs3) (S~\odot~(txs1 ++ txs2)) \\
  % =  (S~\odot~(txs1 ++ txs2)) \odot txs3 = (S~\odot~((txs1 ++ txs2) ++ txs3))
  %
  %  ~\fun{Hom}~S1~S3$
\end{itemize}


\subsection{Category of all ledgers}
\label{sec:all-ledgers}

We first introduce some notions needed to describe the category of all ledgers $\type{Led}$
and maps between them that preserve ledger structure.

Let $\Set_{*}$ denote the category of pointed sets with maps $f \in \fun{Hom} (\{*\} \to S,~\{*\} \to S)$,
such that $({*} \mapsto x \in S, s \in S) \mapsto (* \mapsto f(x), f(s)) \in Q$ for some
set map $f : S \to Q$.

Let $\Monoid$ denote a category of monoids $(M : \Set,~ \mu~:~ \Set \times \Set \to \Set,~ e : M)$
and monoid action- and identity-preserving homomorphisms. We use $(M, \mu, e)$ and $M$
interchangeably here.

Let $\langle \wcard \rangle ~:~\Monoid~\dashv ~\Set~ :~ \mathcal{U}$ be the free-forgetful adjunction
for monoids.

Let us now define a functor $\Gamma$ :

\[ \Gamma : \begin{cases}
  \Set_{*} \times \Monoid \to \Set_{*} \times \Monoid  \\
  (i : \{*\} \to S, M) \mapsto (\{*\} \to S \times \mathcal{U} M, M) \\
  (f \times g : (i : \{*\} \to S, M) \to (\{*\} \to Q, N)) \mapsto
    ((\langle f,~\mathcal{U}~g \rangle, g)  (\{*\} \to (S \times \mathcal{U} M), M) \to (\{*\} \to (Q \times \mathcal{U} N), N))
\end{cases} \]

We now define the natural transformations $\gamma : \Gamma^2 \to \Gamma$ and
$\eta : \mathbb{1}_{\Set \times \Monoid} \to \Gamma$.

Note that

\[ \Gamma (i : \{*\} \to S, M) = (\langle i, \mathcal{U} e\rangle : \{*\} \to S \times \mathcal{U} M, M) \]

and

\[\Gamma^2 (i : \{*\} \to S, M) =
(\langle i, \mathcal{U} e, \mathcal{U} e\rangle : \{*\} \to S \times \mathcal{U} M \times \mathcal{U} M, M) \]

So that we define

\[ \gamma_{(i : \{*\} \to S, M)} : \begin{cases}
   \Gamma^2 (i : \{*\} \to S, M) \to \Gamma (i : \{*\} \to S, M)
  \\
  (* \mapsto (i, \mathcal{U} e, \mathcal{U} e), (s, m_1, m_2), m) \mapsto (* \mapsto (i, \mathcal{U} (\mu e e)), (s, \mathcal{U} (\mu m_1 m_2), m)
\end{cases} \]

\[ \eta_{(i : \{*\} \to S, M)} : \begin{cases}
  (i : \{*\} \to S, M) \to (\langle i, \mathcal{U} e\rangle : \{*\} \to S \times \mathcal{U} M, M)  \\
  (* \mapsto i, s, m) \mapsto (* \mapsto (i, \mathcal{U} e), (s, \mathcal{U} e) m)
\end{cases} \]

The data $(\Gamma, \gamma, \eta)$ is a monad on $\Set_{*} \times \Monoid$. A $\Gamma$-algebra
is made up of $(i : \{*\} \to S, M)$ together with the following operation, induced
by a monoid action $\odot : S \times M \to S$

\[ \odot_{(i : \{*\} \to S, M)} : \begin{cases}
  \Gamma (i : \{*\} \to S, M) \to (i : \{*\} \to S, M) \\
  (* \mapsto (i, \mathcal{U} e), (s, m_1), m) \mapsto (* \mapsto i, (s \odot m_1), m)
\end{cases} \]

We define the category of all ledgers, $\mathcal{L}$, in terms of these structures.
Recall here that the morphisms in the product category with objects $\Set_{*} \times \Monoid$
are those that preserve $*$ in the first coordinate, and monoid structure in the second.

\begin{itemize}
  \item[(i)] Objects : \newline
  $\fun{Obj}~\mathcal{L} = \Set_{*} \times \Monoid$ \newline

  \item[(ii)] Morphisms are a subset of the morphisms of the product morphisms
  between objects of $\Set_{*} \times \Monoid$ which contains only those maps that
  preserve each $\odot_{(i : \{*\} \to S, M)}$  : \newline

  $\fun{Hom}~(i : \{*\} \to S, M)~(i : \{*\} \to Q, N) = \{ f~:~(i : \{*\} \to S, M) \to (i : \{*\} \to Q, N) ~\mid~\\
  ~~~~ \odot_{(i : \{*\} \to Q, N)}~\circ~(\Gamma~ f)~=~ f \circ \odot_{(i : \{*\} \to S, M)} \}$ \newline

  \item[(iii)] Identity :
  $ \odot_{(i : \{*\} \to Q, N)}~\circ~(\Gamma~ \mathbb{1}_{(i : \{*\} \to S, M)})~ \\
  ~~~~~ = \odot_{(i : \{*\} \to Q, N)}~\circ~ \mathbb{1}_{\Gamma~(i : \{*\} \to S, M)} \\
  ~~~~~ = \odot_{(i : \{*\} \to Q, N)} \\
  ~~~~~ = \mathbb{1}_{(i : \{*\} \to S, M)} \circ \odot_{(i : \{*\} \to Q, N)}$ \newline

  \item[(iv)] Composition (we drop the subscript of $\odot$ from here onwards) : \\
  $\forall~f \in \fun{Hom}~(i : \{*\} \to S, M)~(i : \{*\} \to Q, N),~
  g \in \fun{Hom}~(i : \{*\} \to Q, N)~(i : \{*\} \to P, R), \\
  (g \circ f)~\circ~\odot~ = ~g (f \circ \odot) = ~g ~(\odot \circ (\Gamma f)) \\
  ~~~~ = ~g ~(\odot \circ (\Gamma f)) \\
  ~~~~ = ~\odot \circ (\Gamma f) \circ (\Gamma f) \\
  ~~~~ = ~\odot \circ (\Gamma (g \circ f))$ \newline

  \item[(v)] Associativity : inherited from maps in $\Set_{*} \times \Monoid$.

\end{itemize}

\section{Deterministic Ledgers}
\label{sec:determinism}

Determinism in the context of ledgers and ledger state updates refers to the
idea that a user has no control over the order in which submitted transactions
will be applied to a ledger state, what the ledger state will be
when their transactions finally get applied, or even the precise specification
of the update that is being computed. In some cases, randomness,
or differences in rounding or different machines, or certain kinds of
data access can be a source of indeterminism.

Here we attempt to make precise the sort of determinism that assumes that the
function $\ups$ does not itself vary at all, as it would in the case of using
different rounding strategies for computing $\ups~tx~s$ on different machines.
Investigating this source of indeterminism will be part of future work.

Here we give two definitions of determinism and compare them.

\subsection{Order-determinism}
\label{sec:order-det}

\begin{figure*}[htb]
  \emph{Order-determinism constraint on a ledger specification}
  %
  \begin{equation*}
    \begin{array}{l@{~~}l@{~~}ll}
    \fun{orderDetConstraint}_L ~=~ & \forall~(s~\in~\type{ValSt}_L)~(txs~\in~\seqof{\type{Tx}_L}),~\\
    & ~~~~txs'~\in~\fun{Permutation}~txs,~\\
    & ~~~~\fun{validUpdate}_L~s~txs~\neq~\Err~\neq~\fun{validUpdate}_L~s~txs' \\
    & ~~~~\Rightarrow~~\fun{validUpdate}_L~s~txs~=~\fun{validUpdate}_L~s~txs'
    \end{array}
  \end{equation*}
  \caption{Order-determinism}
  \label{fig:order-det}
\end{figure*}

The definition in Figure \ref{fig:order-det} is given in terms of the traces leading
to a particular ledger state. Specifically, it is concerned with the independence of the
ledger state from the order
of the transactions in the trace that has lead to any given state.

\todopv{Randomness as a source of determinism can be accounted for in this definition - reorder
the transactions and you get a different outcome randomly, however,
things like mismatched rounding do not - this is a consensus-level issue?}

\subsection{Stateful-determinism}
\label{sec:state-det}

\begin{figure*}[htb]
  \emph{Arbitrary state-determinism constraint on a ledger specification}
  %
  \begin{equation*}
    \begin{array}{l@{~~}l@{~~}ll}
    \fun{stateDetConstraint} ~=~ & \forall (s~s'~\in~\type{ValSt}_L)~(tx~\in~\type{Tx}_L),~\\
    & ~~~~\fun{updateErr}_L~tx~s~\neq~\Err~\neq~\fun{updateErr}_L~tx~s \\
    & ~~~~\Rightarrow~~\Delta_L~[tx]~s~=~\Delta_L~[tx]~s'
    \end{array}
  \end{equation*}
  \caption{State-determinism}
  \label{fig:state-det}
\end{figure*}

\todopv{make a note here about only the transaction body making the changes,
which is a part of the tx that, if kept constant, makes the same s' from every s
regardless of the changes to the rest of the tx (and produces Err in the same
cases)}

The definition in Figure \ref{fig:state-det} is operational, or local. It describes
the requirement that given any valid ledger state, the change in the ledger state
resulting from the application of a transaction to one state is, in some sense,
the same as the change resulting from the application of that transaction to a
different ledger state, given that in both cases, the transaction is valid.

We will later specify what exactly is the function

\[ \Delta_L~\in~\seqof{(\type{Tx}_L)} \to \type{ValSt}_L \to \Diff_L \]

This function must express \emph{the changes that applying
an update $txs$ to the ledger state $s$ makes}.
What the state-determinism constraint says is that the update (or, change set)
to the state of a state-deterministic ledger is specified uniquely by the transaction list
$txs$ that is being applied, and is independent of the state to which it is applied in the
case that the update is valid in that state.

A definition of $\Delta_L$ making the constraint trivially true is easy to give.
However, in order for $\Delta_L$ to carry the intended meaning, the ledger
itself must admit a certain kind of structure, with which $\Delta_L$ indeed reduces to a
function that simply discards its second argument and returns its first.

In the general case of an arbitrary $L$, the definition of $\Delta_L$ may not have
this property. In fact, it should be very dependent on the specific data structures
of $\type{Tx}_L$ and $\type{State}_L$. A theory of changes and program derivatives
is presented in \cite{changes}. We use this theory to make precise the definition of $\Delta_L$,
and to describe the structure characteristic
of state-deterministic ledgers that admit a trivial $\Delta_L$.

\subsection{Derivative structure in single-ledger categories}

This work gives a framework for defining \emph{derivatives}, where, for a given input,
a derivative maps changes to that input directly to changes in the output.
This approach adheres to some of the key ideas of the usual notion of differentiation
of functions, while adapting them to programs. Other abstract notions of differentiation,
such as presented in \cite{diffrestcats}, impose additional constraints on
the functions being differentiated, such as the definition of addition of functions.
The theory of changes outlined in \cite{changes} draws inspiration from an
abstract notion of derivation to deal with changes to data structures.

One noteworthy difference between the differentiation presented in the two papers is
the type of the derivative function. A differential category derivative of a
function $f : X \to Y$ has the type

\[ \type{D}[f] : X \times X \to Y \]

While the type of a derivative of a program $p : A \to B$ is

\[ \fun{diff}~p : A \to \Diff~A \to \Diff~B \]

The reason for this is that to define a coherent and applicable theory of changes
to a data structure, we want to have a special type $\Diff~A$ for each
structure $A$ that can express changes to $A$. Additional structure required
for functional differentiation allows for a theory wherein the changes to a term of a type
can be expressed by another term of that same type.

Now that we gave some justification for discussing the type of changes to a ledger
state, we can explore what that looks like. The change type operator, $\Diff$, introduced
in \cite{changes}, along with its behaviour and the constraints on it, is specified in Figure \ref{fig:diff}.
Here we tailor the definition to a single ledger specification, so that it only
describes the change type of the ledger state type for a given ledger specification.

In the work mentioned above, all functions are total, as they are applied only
to the domains (sets) in which they are valid.
However, in our model, the partiality of transaction application
to the ledger state object is integral to the discussion of determinism.
Disallowing the application of certain updates to certain states is what
allow us to guarantee deterministic updates in the valid update cases --- so, we
must be able to reason about the error cases.

Moreover, we want to be able to discuss
what it means for a particular category of valid ledger states and maps between them
to enjoy differential structure even if applying certain changes to a given
state results in a failed ($\Err$) update.

For this reason, we additionally make the change that only some changes are permitted, i.e. the
ones that lead to a valid ledger state, so that $\fun{applyDiff}$ returns
a error-type.

\begin{figure*}[htb]
  \emph{$\Diff$ Type Accessors}
  %
  \begin{equation*}
    \begin{array}{r@{~\in~}l@{~~~~}lr}
      \type{Spec}_D
      & \Type
      & \text{Accessor for the type of the ledger specification}
      \nextdef
      \type{State}_D
      & \Type
      & \text{Accessor for the type of the valid ledger state, $\type{State}_D = \type{ValSt}_{\type{Spec}_D}$}
      \nextdef
      \Diff_D
      & \Type
      & \text{Accessor for the change type of the ledger state}
    \end{array}
  \end{equation*}
  \emph{$\Diff$ functions}
  %
  \begin{equation*}
    \begin{array}{r@{~\in~}l@{~~~~}lr}
      \fun{applyDiff} & \type{State}_D \to \Diff_D \to \type{State}_D &
      \text{Apply a set of changes to a given state}
      \nextdef
      \fun{extend} & \Diff_D \to \Diff_D \to \Diff_D &
      \text{Compose sets of changes}
      \nextdef
      \fun{zero} & \Diff_D  &
      \text{No-changes term}
    \end{array}
  \end{equation*}
  \emph{$\Diff$ constraints}
  %
  \begin{align*}
      & \fun{zeroChanges} ~\in~\forall~s, \fun{applyDiff}~s~\fun{zero}~=~s  \\
      & \text{Applying the zero change set results in no changes}
      \nextdef
      &\fun{applyExtend} ~\in~ \forall~s~txs1~txs2,~\fun{validUpdate}~s~txs1~\neq~\Err~\neq~\fun{validUpdate}~s~txs2, \\
      &~~~~\fun{applyDiff}~s~ (\fun{extend}~txs2~txs1)~=~\fun{applyDiff} (\fun{applyDiff}~s~txs1)~txs2 \\
      & \text{If both change sets are valid, composing them gives the same result as applying them in sequence}
  \end{align*}
  \caption{Specification for a change type $D \in~\Diff$}
  \label{fig:diff}
\end{figure*}

Now, we want to introduce the idea of taking and evaluating derivatives in
a single-ledger category for a given ledger $L~\in~\LS$, see Figure \ref{fig:diff-cat}.

\begin{figure*}[htb]
  \emph{$\DCat$ Type Accessors}
  %
  \begin{equation*}
    \begin{array}{r@{~\in~}l@{~~~~}lr}
      \type{Spec}_{CD}
      & \Type
      & \text{Accessor for the type of the ledger specification}
      \nextdef
      \type{State}_{CD}
      & \Type
      & \text{Accessor for the type of the valid ledger state, $\type{State}_{CD} = \type{ValSt}_{\type{Spec}_{CD}}$}
      \nextdef
      \type{Diff}_{CD}
      & \Type
      & \text{Accessor for the $\Diff$ type of the ledger state}
      \nextdef
      \type{DerType}_{CD}
      & \Type
      & \text{Accessor for the type representing derivative functions}
    \end{array}
  \end{equation*}
  \emph{$\DCat$ functions}
  %
  \begin{equation*}
    \begin{array}{r@{~\in~}l@{~~~~}lr}
      \fun{takeDer} & \seqof{(\type{Tx}_{\type{Spec}\_{CD}})} \to \type{DerType}_{CD} &
      \text{Take the derivative of a transaction}
      \nextdef
      \fun{evalDer} & \type{DerType}_{CD} \to \type{State}_{CD} \to \Diff_D \to \Diff_D &
      \text{Calculate the diff value of a derivative at $(s,~ds)$}
    \end{array}
  \end{equation*}
  \emph{$\DCat$ constraints}
  %
  \begin{align*}
      &\fun{derivativeConstraint} ~\in~ \forall~s~ds~txs,\\
      &~~~~\fun{validUpdate}_L ~(\fun{applyDiff}~ ds~ s) ~txs~\neq~\Err, \\
      &~~~~\fun{applyDiff} ~(\fun{evalDer} ~(\fun{takeDer}~ txs) ~s~ ds)~ (\fun{validUpdate}_L~ s ~txs)~\neq~\Err, \\
      &~~~~\fun{validUpdate}_L ~(\fun{applyDiff}~ ds~ s) ~txs ~= \\
      &~~~~~~~~ \fun{applyDiff} ~(\fun{evalDer} ~(\fun{takeDer}~ txs) ~s~ ds)~ (\fun{validUpdate}_L~ s ~txs) \\
      & \text{If both change sets are valid, composing them gives the same result as applying them in sequence}
  \end{align*}
  \caption{Structure $DC~\in~\DCat$ for a data-differentiable category }
  \label{fig:diff-cat}
\end{figure*}

\todopv{Should evalDer ever produce an Err ?? txs (ds s) = (f s ds txs) (txs s) }

\subsection{Instantiating derivative structure}

\begin{figure*}[htb]
  \emph{Data-differentiable structure}
  \begin{equation*}
    \begin{array}{l@{~\leteq~}l@{~~~~}lr}
      \Diff_L~&~\seqof{(\type{Tx}_L)} \\
      \fun{applyDiff}~&~\fun{flip}~\fun{validUpdate}_L \\
      \fun{extend} ~&~\fun{flip}~(++) \\
      \fun{zero}~&~[]
    \end{array}
  \end{equation*}
  \emph{$\DCat$ structure instantiation $\DCat_{constDer\_L}$}
  \begin{equation*}
    \begin{array}{l@{~\leteq~}l@{~~~~}lr}
      \type{DerType}_L~&~\Nt \\
      \fun{takeDer}~txs~&~\Nt \\
      \fun{evalDer}~txs~s~ds~&~\begin{cases}
        ds & \text{ if } \fun{validUpdate}_L~s~ds~\neq~\Err \\
        \Err & \text{ otherwise}
      \end{cases} \\
    \end{array}
  \end{equation*}
  \caption{Instantiation of data-differentiable structure in $\type{SLC}_L$ }
  \label{fig:diff-cat-inst}
\end{figure*}

We give one way to instantiate the data-differentiable structure in a single-ledger
category $\type{SLC}_L$ in Figure \ref{fig:diff-cat-inst}, which we will denote
$\DCat_{constDer\_L}$.

This instantiation of the $\Diff$ structure is common to all $\type{SLC}_L$ categories,
and trivially satisfies the constraints in Figure \ref{fig:diff}.
The type representing
changes to the (valid or $\Err$) ledger state, $\Diff_L$, is $\seqof{(\type{Tx}_L)}$
because lists of transactions is exactly the data structure that represents
changes to the state. Applying changes is therefore represented by transaction application
$\fun{validUpdate}_L$. Extending one change set with another is concatenation of
the two transaction lists, and the empty change set is the nil list.

The $\DCat_{constDer\_L}$ structure we specified, however, is not necessarily
commonly admissible as a $\DCat$ instantiation in all ledgers,
as it does not guarantee that the $\fun{derivativeConstraint}$ will always be satisfied.
We will now discuss the relation of the specification we give, the satisfiability of
the derivative constraint, and both versions of the determinism definition.

\subsection{$\Delta$ and derivation structure specification in ledgers}

In this section we present and justify our choice of $\Delta_L$ definition, as well
as use of $\DCat_{constDer\_L}$ structure.

The idea of derivation, as presented in \cite{changes}, is that,
given a function $f$ and an input $s$ to that function, takes a change set $ds$ to another change set $ds'$
such that the changes $ds'$ are consistent with the changes of the original $ds$,
but done after $f$ has been applied --- to the new state output by $f~s$.

Recall that, intuitively, the definition of \emph{state-determinism} conveys that the changes
a transaction makes \emph{are specified entirely within the transaction itself}.

\todopv{this needs more explanation about the connection between Delta and the
derivative constraint and evalDer being a proj function}

The derivative constraint then reduces to, for all $s,~txs,~ds$,

\[\text{(a)~:~~~} \fun{validUpdate}_L~ (\fun{validUpdate}_L~ ds~ s) ~txs~=~\fun{validUpdate}_L ~ds~(\fun{validUpdate}_L~ s ~txs) \]

whenever $\Err$ is not produced by any computation. In particular,

\[ \forall~s,~tx,~\fun{evalDer} ~(\fun{takeDer}~ []) ~s~ [tx] ~=~[tx] \]


\subsection{Order-determinism, state-determinism, and derivation.~}
\label{sec:od-sd-d}
We can prove that if a ledger $L$ is order-determinismic, it admits data-differentiable
category structure with constant derivaties.

\begin{theorem}
  Suppose $L$ is a ledger, and $\DCat_{constDer\_L}$ structure is as specified in Figure \ref{fig:dcat-pf}.
  Then,

  \[ \fun{orderDetConstraint}_L ~\Rightarrow~\type{SLC}_L~\in~\DCat_{constDer\_L} \]

  \label{theo:dcat-pf}
\end{theorem}

\begin{proof}
  For an arbitrary $s,~txs,~ds$, we can instantiate the variables in the
  order-determinism definition in Figure \ref{fig:order-det} by

  \[ \text{(b)~:~~~}\fun{orderDetConstraint}~s~(\fun{exted}~txs~ds)~(\fun{extend}~ds~txs) \]

  Since

  \[ \fun{extend}~ds~txs~\in~\fun{Permutation}~(\fun{extend}~txs~ds) \]

  Whenever (b) holds for $s,~txs,~ds$, it immediately follows that so does $\fun{derivativeConstraint}$ .
\end{proof}

We can also prove that assuming a leger admits constant derivaties, it is
state-deterministic.

\begin{theorem}
\label{the:state-det}
  Suppose for a given ledger $L$ with $\DCat$ structure instantiated as in
  Figure \ref{fig:diff-cat-inst}, the following hold :

  \begin{itemize}
    \item[(i)] \[ \forall~s,~txs,~tx,~\fun{evalDer} ~(\fun{takeDer}~ txs) ~s~ [tx]~=~[tx] \]
    \item[(ii)] $\fun{derivativeConstraint}_L$
    \item[(iii)] \[ \Delta_L~[tx]~s~=~\fun{evalDer} ~(\fun{takeDer}~ []) ~s~ [tx]) \]
  \end{itemize}

  It follows that $\fun{stateDetConstraint}_L$ holds.

\end{theorem}

\begin{proof}
  To prove this, observe that in a category $\type{SLC}_L$ that admits $\DCat_{constDer\_L}$ structure
  as specified in Figure \ref{fig:diff-cat-inst}, we get that for any $s,~s'$ and $tx$
  valid in $s$ and $s'$,

  \[ \Delta_L~[tx]~s~=~\fun{evalDer} ~[] ~s~ [tx]~=~[tx]~=~\fun{evalDer} ~[] ~s'~ [tx])~\Delta_L~[tx]~s' \]

  Which proves the result.

\end{proof}






If constant derivatives are admissible in a ledger $L$, we can conclude that
it is an order-deterministic ledger as well.

\begin{theorem}
\label{the:state-det}
  Suppose for a given ledger $L$ with $\DCat$ structure instantiated as in
  Figure \ref{fig:diff-cat-inst}, the following hold :

  \begin{itemize}
    \item[(i)] \[ \forall~s,~lstx,~tx,~\fun{evalDer} ~(\fun{takeDer}~ lstx) ~s~ [tx] \]
    \item[(ii)] $\fun{derivativeConstraint}_L$
  \end{itemize}

  It follows that $\fun{orderDetConstraint}_L$ holds.

\end{theorem}

We use shorthand here

\[ (s~ds)~:=~\fun{validUpdate}_L~ s~ ds \]

\[ \fun{evalDer} ~txs1 ~s~ tx~ \leteq ~\fun{evalDer} ~(\fun{takeDer}~ txs1) ~s~ tx \]

\begin{proof}

Suppose we have a ledger where defining $\DCat_{constDer\_L}$ structure as in \ref{fig:diff-cat-inst}
satisfies the $\fun{derivativeConstraint}$.

We use strong induction on the length of $txs~\in~[\Tx]$ to prove that for any $s$,

\[ \forall~txs'~\in~\type{Permutation}~txs,~(s~txs)~\neq~\Err~\neq~(s~txs')~\Rightarrow~ (s~txs)~=~(s~txs') \]

Base case, $txs = []$ is trivial.
Induction step : suppose that for some $n$, for any $txs$ with $\fun{len}~txs~~\leq~n$,
the above claim is holds. We want to show that it holds for $n+1$.

We can say that for any decomposition $txs~=~txs1~++~txs2$, for some two lists $txs1,~txs2$,

\[ (s~tx)~txs ~=~((s~tx)~txs1)~txs2 \]

And, if

\[((s~tx)~txs1)~txs2~\neq~\Err~\neq~((s~txs1)~(\fun{evalDer} ~txs1 ~s~ tx))~txs2 \]

We get that, by the $\fun{derivativeConstraint}$,

\[ ... ~=~ ((s~txs1)~(\fun{evalDer} ~txs1 ~s~ tx))~txs2~=~((s~txs1)~tx)~txs2 \]

Now, $(s~txs1)$ is independent of the order of $txs1$ by the strong induction hypothesis,
as is $((s~txs1)~tx)~txs2$ on the order of $txs2$ (or $txs1$), and
$(s~txs)$ on the order of $txs$.

We can conclude that inserting $tx$ anywhere in the list $txs$ gives the same
state $((s~txs1)~tx)~txs2$ or $\Err$. That is,
any permutation of the list of length $n+1$, $txs1~++~[tx]~++~txs2$,
will result in the same state, or an $\Err$ as well. This proves the
result for permutations of lists of arbitrary length.

\end{proof}


So, if we choose to define the ledger changes $\Delta$ as in \ref{the:state-det},
a ledger is both state- and order-deterministic exactly when it admits
the following definition of a derivative of a single transaction (such that
it satisfies the $\fun{derivativeConstraint}$) :

\[\fun{evalDer} ~ls ~s~ [tx] ~=~ [tx] \]

\todopv{is it true that evalDer indep. of function => it is indep. of state too, and vice-versa?}


\subsection{Update Determinism.~}
\label{sec:det-examples}

\subsection{Examples.~}
\label{sec:det-examples}

In some of the following examples, we use an orthogonal but complementary context
to determinism --- the notion of \emph{replay protection}.

\begin{definition}
  In a ledger with \emph{replay protection},

  \[ \forall~lstx,~tx,~tx~\in~lstx ~ \Rightarrow~\fun{validUpdate}_L~\fun{initState}~(lstx~++~[tx])~=~\Err \]
\end{definition}

This is a valuable property to have to avoid adversarial or accidental replaying of transactions.

\begin{itemize}
  \item[(i)] \emph{UTxO ledger}. A basic UTxO ledger $UTxO$ with the constraint in its $\ups_{UTxO}$ function
  requiring that $\fun{nonEmpty}~(\fun{inputs}~tx)$ provides replay protection since
  each output can only be consumed once, preventing any other (or the same) transaction consuming that
  output again, and therefore from being reapplied.

  A UTxO ledger is also deterministic. Note that we assume here that there are no hash collissions.
  A hash collission possibility undermines both determinism and replay protection.\\

  \item[(ii)] \emph{Account ledger with value proofs.} A ledger $AP$ where the state consists of a
  map $\State_{AP}~\leteq~\type{AccID}~\mapsto~\type{Assets}$, and $\Tx_{AP}~\leteq~(\State_{AP},~\State_{AP})$.
  If for a given state and transaction $s,~tx$, $\fun{fst}~tx~\subseteq~s$, the
  update function outputs $(s~\setminus~\fun{fst}~tx)~\cup~(\fun{snd}~tx)$. Presumably,
  there will also be a preservation of value condition imposed on the asset total.

  This ledger provides no replay protection --- one can easily submit transactions
  that keep adding and removing the same key-value pairs.

  Note here that a regular account-based ledger is also deterministic, but not when
  smart contracts or case analysis is added (eg. the $\ups$ allows for transfer
  for a quantity $q$ \emph{or} any amount under $q$ if $q$ is not available). This
  value-proof feature makes it possible to enforce determinism even in such
  conditions of multiple control flow branches. Account-based ledgers also, unlike
  the UTxO model, do not offer a simple, space-efficient solution to replay protection.
  Ethereum does offer it, but it is somewhat convoluted, see EIP-155. \\

  \item[(iii)] \emph{Account ledger with value proofs and replay protection}. It is easy to add
  replay protection to the previous example. For example, by extending the data
  stored in the state,

  \[ \State_{APreplay}~\leteq~([\Tx_{AP}],~\type{AccID}~\mapsto~\type{Assets}) \]

  The update function will then have an additional check that a given $tx~\notin~\fun{txList}~s$. \\

  \item[(iv)] \emph{Non-deterministic differentiable ledger.} It is possible to have a ledger
  that is differentiable, has replay protection, but is not deterministic. Let us
  extend the above state with a boolean,

  \[ \State_{APB}~\leteq~(\Bool,~[\Tx_{APB}],~\type{AccID}~\mapsto~\type{Assets}) \]

  \[ \Tx_{APB}~\leteq~((\Bool,~\Bool),~(\State_{AP},~\State_{AP})) \]

  And specify $\ups~s~=~s'$ in a way that the boolean of the state $s$ is updated in $s'$ whenever
  the first boolean in the transaction matches the one in the state :

  \[ \fun{bool}~s' ~\leteq~ \begin{cases}
     \fun{snd}~(\fun{bool}~tx) & \text{ if } \fun{fst}~(\fun{bool}~tx)~==~\fun{bool}~s \\
     \fun{bool}~s & \text{ otherwise}
  \end{cases} \]

  We can define differentiation for such a ledger in a way that allows switching the order of application
  of functions $ds$ and $txs$ by using re-interpreting the changes $txs$ makes when $ds$ is applied
  first so that the $\fun{derivativeConstraint}$ is satisfied :

    \[\fun{evalDer} ~txs ~s~ ds ~\leteq~ ds++ [txFlip] \]

    where

    \[ txFlip~=~((\fun{bool}~(s~txs),~\fun{bool}~(s~ds~txs)),~\Nt) \]

    The $\fun{evalDer}$ function cannot be defined to be a projection of the third
    coordinate, as this does not satisfy the constraint. It is straightforward to change the $\ups$
    function of the $APB$ ledger
    such that a valid derivative function can be a simple projection, and the ledger - deterministic. The $\ups$ function must
    produce an $\Err$ instead of
    allowing the state update to take place even if $\fun{fst}~(\fun{bool}~tx)~==~\fun{bool}~s$ does not hold.


\end{itemize}

\section{Update-deterministic ledgers}
\label{sec:update}

Determinism, both state- and order- (SD and OD, for short), specifies what it means for a ledger to behave
predictably under an update caused by a transaction in the face of a lack of information about
what the state might be at any given time, or what transactions happen to have
been processed before the transaction doing the update in question.
This approach to specifying determinism focuses on allowing only certain transactions
to be processed.

Here we propose another angle from which to investigate determinism,
which instead focuses on the properties of the $\ups$ function itself, as well
as the state and transaction structure.

A ledger $L$ is \emph{without inverses} whenever

  \[ \forall~lts~s,~(s~\var{lts})~=~s~\Leftrightarrow~\var{lts}~=~[] \]

If a ledger has replay protection, it is without inverses.
Replay protection precludes inverses, since having
inverses implies that applying a list of transactions is possible more than once,

\[ (s~\var{lts}~\var{lts})~=~(s~\var{lts})~=~s \]

A ledger $L$ is \emph{update-deterministic} (or, UD) whenever it has
\emph{consistent transaction application}, defined in Figure \ref{fig:consistent}

\begin{figure*}[htb]
  \begin{equation*}
    \begin{array}{l@{~~}l@{~~}ll}
    \fun{updateDetConstraint}_L ~=~ & \forall~s~tx~tx',~\\
    & ~~~~ (\Err~\neq~(s~\var{tx})~=~(s~\var{tx'})~\neq~\Err~\Rightarrow~\\
    & ~~~~\forall~s',~(s'~\var{tx})~\neq~\Err~\neq~(s'~\var{tx'}) \\
    & ~~~~~~~~\Rightarrow~(s'~\var{tx})~=~(s'~\var{tx'}))
    \end{array}
  \end{equation*}
  \caption{Consistent transaction application}
  \label{fig:consistent}
\end{figure*}


\begin{theorem}
  If a ledger $L$ is order-deterministic, it is update-deterministic.
\end{theorem}

\begin{proof}
   \todopv{Prove this! I think it's true}
\end{proof}

UD is a weaker condition. It captures the idea that while the transaction order
cannot be changed, there are no additional transaction application computations done
using the order (see example (i) and (ii) below). In a sense, some order-dependent
internal indexing takes place when a transaction is applied. If this indexing is not
later inspected by other computations on-chain, there is no issue. However, building
applications than do inspect this indexing data off-chain is bad.

This definition is also representative of the idea that state-determinism attempted to
capture (but ended up defaulting to OD), that a list of transactions makes the same changes to the ledger no matter
what state they are applied to.

\todopv{Give example of pointer addresses on Cardano. The process of building the map
of pointer addresses as transactions update the delegations map could itself be
UD (it is just an internal indexing process), but we notice that in fact, reward
calculations are done using this pointer address
map, so UD is broken.}

\textbf{Examples : } \\

\begin{itemize}
  \item[(i)] $\Tx~ = ~\N$, $\State~ = ~[\N]$, and the update function appends the number in a $tx$ to the list. This ledger
  is update-deterministic.

  \item[(ii)] $\Tx~ = ~\N$, $\State~ = ~{~ n~\in~\N~}^*$, where the update function includes the number in the $tx$ in the multiset
  (${...}^*$ is the notation for multiset). This ledger is OD

  \item[(iii)] $\Tx~ = ~\Bool$, $\State~ = ~\Bool$, where the $tx$ flips the bit in the state if they match. This ledger is not
  deterministic.

  \item[(iv)] $\Tx~ = ~\Bool$, $\State~ = ~\Bool$, where the update function is XOR. This ledger is OD (and so, probably UD),
  but has inverses.
\end{itemize}

\subsection{UD and derivatives}

Recall that in Section \ref{sec:od-sd-d} we discussed the relation between derivation
in the ledger and OD. We can also make a claims about UD and derivation.

\textbf{Claim : } A derivative of a UD ledger must be independent of the state,
but can still depend on the set of changes $ds$ applied before or after the
transaction list $txs$, of which we are calculating the derivative.
Recall the constraint on valid derivatives (using shorthand notation) :

\[ \fun{derivativeConstraint} ~\in~ \forall~s~ds~txs, \]

\[ ~((s~ds)~txs)~\neq~\Err~\neq~((s~txs) ~(txs ~s~ ds)')~\Rightarrow~((s~ds)~txs)~=~((s~txs) ~(txs ~s~ ds)') \]

The $\fun{updateDetConstraint}_L$ then tells us that

\[ \forall~s_1~s_2,~ (s~(txs~++~(txs~s_1~ds)'))~=~(s~(txs~++~(txs~s_2~ds)')) \]

And therefore the derivative $(txs~s~ds)'$ must not be dependent on $s$.

\todopv{can we say anything about this in the other direction? ie. derivative is
independed of s implies ... ?}


\subsection{Implementation of an OD $\ups$ function}

\todopv{clean this up!!}

This section we consider only ledgers that can be represented as partial products, ie.
products $P~\times~Q$ that have functions into them from some type $R$

\[ \langle~\fun{p},~\fun{q}~\rangle~:~R~\to~P~\times~Q \]

return $\Err$ whenever \emph{either} $\fun{p}$ \emph{or} $\fun{q}$ returns $\Err$,
as well as possibly in other cases.

For example, the UTxO set is (maybe?) isomorphic to a (probably finite) product of Maybe-outputs,
where each element of the product object looks like $(o_1~,~ .... ,~ o_k)$, where each spot in the
tuple corresponds to an input.

We conjecture that ledgers that have a partial product decomposition such that
the $\ups$ function can be expressed as a \emph{projection} are OD, so that

\[ \ups ~=~\pi_{TxS} \]

Note that this does not mean that this is simply a projection onto the second
coordinate. It rather means that all data needed to construct the
new state is stored in either the old state or transaction, and no computation outside
of projection is allowed.

Projections commute (except when $\Err$ occurs).

% inputs \in \Pi_{i~\in~\type{InputType}}, outputs
%
% inputs |-> outputs
%
% pi_{TxS} ((inputs_tx, outputs_tx), inputs |-> outputs) = \Pi_{i~\in~(inputs ~\setminus~ inputs_tx)} \pi
%

If we have transaction

\[ tx = (inputs~=~(txid, 1),~(txid, 2);~outputs~=~(txout_1,~txout_2)) \]

\begin{figure*}[htb]
  \emph{Projections}
  %
  \begin{equation*}
    \begin{array}{l@{~:~}l@{~~}ll}
    \pi_{TxS} & (\Tx,~\State) \to \State  \\
    & \text{projects data that will be in the state after applying $\ups$ }
    \end{array}
  \end{equation*}
  \emph{UTxO example}
  %
  \begin{equation*}
    \begin{array}{l@{~~}l@{~~}l@{~~}l@{~~}l@{~~}l@{~~}l@{~~}l@{~~}l}
      & txid_1 & txid_2 & txid_3 & ...\text{all possible $txid$s}... & txid_k   \\
      & v_1 & \Nt & v_3 & ...\text{values corresponding to above $txid$s}... & \Nt   \\
    \end{array}
  \end{equation*}
  \emph{View of UTxO by tx}
  %
  \begin{equation*}
    \begin{array}{l@{~~}l@{~~}l@{~~}l@{~~}l@{~~}l@{~~}l@{~~}l@{~~}l}
    & txid_1 & txid_2~(=~(txid, 1)) & txid_3~=(~(txin_1)) & ...\text{all possible $txid$s}... & txid_k~(=~(txid, 2))   \\
    & v_1 & txout_1 & v_3 & ...\text{values corresponding to above $txid$s}... & txout_2
    \end{array}
  \end{equation*}
  \caption{Transaction and state composition of OD ledgers}
  \label{fig:ud-comp}
\end{figure*}

%\section{Multi-set Specified Ledgers}
\label{sec:multiset}

In Section \ref{sec:determinism} we discussed the impact of re-ordering transactions,
and the possibility of reaching different states by applying the same multi-set
of transactions to a given state. We will now consider when, if ever, it should
be possible to reach a given state with distinct multi-sets.

\begin{figure*}[htb]
  \emph{Order-determinism constraint on a ledger specification}
  %
  \begin{equation*}
    \begin{array}{l@{~~}l@{~~}ll}
    \fun{orderDetConstraint}_L ~=~ & \forall~(s~\in~\type{ValSt}_L)~(txs~\in~\seqof{\type{Tx}_L}),~\\
    & ~~~~txs'~\in~\fun{Permutation}~txs,~\\
    & ~~~~\fun{validUpdate}_L~s~txs~\neq~\Err~\neq~\fun{validUpdate}_L~s~txs' \\
    & ~~~~\Rightarrow~~\fun{validUpdate}_L~s~txs~=~\fun{validUpdate}_L~s~txs'
    \end{array}
  \end{equation*}
  \caption{Order-determinism}
  \label{fig:order-det}
\end{figure*}

\section{Threads}
\label{sec:threads}

We can identify a particular kind of a lens-like structure which may arise in a ledger specification,
which we call a \emph{thread}, and denote $T \in \Th$. More specifically,

\begin{definition}
  A \emph{thread} $T \in \Th$ in a ledger specification $L$ is a tuple of
  \begin{itemize}
    \item[(i)] an underlying type $T$,

    \item[(ii)] a projection function
    \[ \fun{proj}_T ~\in~\type{State}_L \to T^? \]

    That is surjective, with a right inverse $\fun{inj}_T$ :

    \[ \fun{proj}_T~\circ~\fun{inj}_T ~=~\fun{id}_{T^?} \]

    \item[(iii)] an update function $\fun{updateT}_T$ is such that
    for all $tx,~s$,

    \[ \fun{updateT}_T~tx~s~(\fun{proj}_T~s) = \Err~ \Rightarrow~ \ups~tx~s~=~ \Err \]

    If $\ups~tx~s~\neq~ \Err$,

    \[ \fun{updateT}_T~tx~s~(\fun{proj}_T~s) = \fun{proj}_T~(\ups~tx~s)\]
  \end{itemize}
\end{definition}

Note that the projection function is not from a valid state, but rather from an
arbitrary state. This function can be extended to a map between dependent types :

\[ \fun{depProj}_T ~\in~\type{State}_L \to T_{val} \]

where

\[ T_{val}~\leteq~\{ ~(s,~t)~ \mid~\fun{proj}_T~s~=~t~\wedge~s~\in~\type{ValSt}_L~ \} \]

Note also that the projection function may return $\Nt$, which implies that all threads
in a ledger are non-persistent. While this is not true for most realistic ledgers
(eg. the UTxO set thread always exists on the ledger, even if it is empty), a ledger specification may
restrict the collection of valid states to only those where persistent threads
are never projected to $\Nt$.

The function $\fun{updateT}$ which applies a transaction-state pair to a $t \in T$
is defined in terms of the $\ups$ function in Figure \ref{fig:threads}.

Recall the one-object category than the one-object category $\type{SLC}_L$.

We construct the category as follows :



 where each $v \in T$ is an image of some $s \in S$.
  and the hom-set of $T \to T$ is given by pairs of type $(\type{ValSt}_{L},~\seqof{(\type{Tx}_{L})})$.

  The update function used to apply a list of transactions $(s,~txs)$ is given by
  folding $\fun{updateT}$ over $txs$ at $s$.
  That is, maps in this category are specified by both the transactions being applied to the ledger,
  as well as the ledger state itself.

  \todopv{Can we do pointed pairs here (i.e. $(t, s)$) instead as objects, rather than just the thread?}


Note here that the implication is one direction --- if the $\fun{updateT'}_T$
does not produce an error, the $\ups$ function can still produce an error applied
to the same transaction and state. That is, intuitively, a thread update function
only cares that the transaction is attempting to
update the thread data badly, and not about what it does to the rest of the ledger.

Since a $\Err$ in the output of the $\fun{updateT'}_T$ function implies in an error
in the $\ups$ function applied to the same state and transaction,
$\fun{updateT'}_T$ is said to \emph{emit constraints} on the ledger update
function.



If we consider individual stateful smart contracts as threads, we again see that
the update of the total UTxO set (or the account-based ledger's state) require
that some general rules must be satisfied, while each smart contract will
have its own constraints that it may place on the transaction performing its
state update.

\begin{itemize}
  \item[(i)] A \emph{temporary} thread $T$ is one that may not appear in some valid ledger
  states of a ledger specification $\type{Spec}~T = L$, i.e.

  \[ \exists~~s~\in~\type{ValSt}_L,~\fun{proj}_T~s~=~\Nt \]

  A permanent thread is one that is not temporary.

  \item[(ii)] A \emph{stateful} thread $T$ is one where the projection function,
  for some pair of ledger states, gives two different non-$\Nt$ outputs.
  \[ \exists~~s,~s'~\in~\type{ValSt}_L,~\Nt~\neq~\fun{proj}_T~s~\neq~\fun{proj}_T~s'~\neq~\Nt \]

  \item[(iii)] We say that the domain of a thread $T$ with the update function
  $\fun{updateT}_T$ \emph{constrains} the
  update of the ledger state when for some $tx$,

  \[ \fun{dom}_T~(tx,~s)~\neq~\fun{id}_T \]

  \item[(iv)] Threads $T$ and $U$ in a ledger specification $L~=~\type{Spec}_T~=~\type{Spec}_U$
  are said to be \emph{independent} when any change by a transaction $tx$ to the state of both threads can
  be instead implemented by two lists of transactions $txs1, txs2$, where

  \begin{itemize}
    \item applying the
    list $txs1$ to the ledger state updates one of the threads to have the same state as after applying
    $tx$ to the ledger while keeping the state of the other thread constant, and
    \item applying the list $txs2$ to the resulting ledger state updates the state of the
    other thread to be the same as after applying $tx$.
  \end{itemize}

  \[ \forall~tx~s~\in~\type{ValSt}_L,~~\fun{validTx}_L~tx~s~\Rightarrow~~\exists~~txs1~txs2, \]
  \[ \fun{proj}_{T}~(\fun{updateErr}_{L}~tx~s)~=~\fun{proj}_{T}~(\fun{computeState}_L~s~(txs1++txs2)) \]
  \[ \wedge~ \fun{proj}_{U}~(\fun{updateErr}_L~tx~s)~=~\fun{proj}_{U}~(\fun{computeState}_L~s~(txs1++txs2)) \]
  \[ \wedge \]
  \[ \fun{proj}_{T}~(\fun{computeState}_L~s~txs1)~=~\fun{proj}_{T}~(\fun{updateErr}_L~tx~s)  \]
  \[ \wedge~ \fun{proj}_{U}~(\fun{computeState}_L~s~txs1)~=~\fun{proj}_{U}~s \]

  \item[(iv)] Threads $T$ and $U$ in a ledger specification $L~=~\type{Spec}~T~=~\type{Spec}~U$
  are said to be \emph{separateable by message passing} when there exist threads $T'$ and $U'$
  such that
  \[ \forall~tx~s~\in~\type{ValSt}_L,~~\fun{validTx}_L~tx~s \]
  ......
\end{itemize}

\todopv{when can two dependent threads be made separable?}

\subsection{Deterministic threads}

\todopv{Does this work?}

See Figures \ref{fig:order-thread}, \ref{fig:update-thread}. UD defined in \ref{sec:update}.

\begin{figure*}[htb]
  \emph{Order-determinism for a thread $T$ in ledger $L$}
  %
  \begin{equation*}
    \begin{array}{l@{~~}l@{~~}ll}
      \forall~(s~\in~\type{ValSt}_L)~(txs~\in~\seqof{\type{Tx}_L}),~\\
      & ~~~~txs'~\in~\fun{Permutation}~txs,~\\
      & ~~~~\fun{validUpdate}_L~s~txs~\neq~\Err~\neq~\fun{validUpdate}_L~s~txs' \\
      & ~~~~\Rightarrow~~\fun{proj}_{T}~(\fun{validUpdate}_L~s~txs)~=~\fun{proj}_{T}~(\fun{validUpdate}_L~s~txs')
    \end{array}
  \end{equation*}
  \caption{Order-determinism for a single thread}
  \label{fig:order-thread}
\end{figure*}

\begin{figure*}[htb]
  \emph{Update-determinism for a thread $T$ in ledger $L$}
  %
  \begin{equation*}
    \begin{array}{l@{~~}l@{~~}ll}
      % \forall~s~tx~tx',~\\
      % & ~~~~ (\Err~\neq~(s~\var{tx})~=~(s~\var{tx'})~\neq~\Err~\Rightarrow~\\
      % & ~~~~\forall~s',~(s'~\var{tx})~\neq~\Err~\neq~(s'~\var{tx'}) \\
      % & ~~~~~~~~\Rightarrow~\fun{proj}_{T}~(s'~\var{tx})~=~\fun{proj}_{T}~(s'~\var{tx'}))
    \end{array}
  \end{equation*}
  \caption{Consistent transaction application (UD) for a single thread $T$}
  \label{fig:update-thread}
\end{figure*}

\section{Blocks, transactions, and threads}
\label{sec:blockchain}

We can use the notion of threads to represent the relationship between blocks
and transactions.
Let us consider an illustrative example, which we will afterwards generalize to
formally state the structure most blockchains are instances of.
Consider the types are as specified in Figure \ref{fig:block-thread}.

\begin{figure*}[htb]
  \emph{Block example types}
  %
  \begin{equation*}
    \begin{array}{r@{~=~}l@{~~~~}lr}
      \Value
      & \Z
      & \text{The ledger's currency} \\
      \Ix
      & \N
      & \text{Index type} \\
      \TxIn
      & (\Tx, \Ix)
      & \text{Transaction input} \\
      \TxOut
      & \Ix~\mapsto~\Value
      & \text{Transaction outputs} \\
      \UTxO
      & \TxIn~\mapsto~\Value
      & \text{UTxO} \\
      \type{UTxOTx}
      & ([\TxIn],~\TxOut,~\Slot)
      & \text{Transaction} \\
      \type{Block}
      & (\Block^?,~[\type{UTxOTx}])
      & \text{Block} \\
      \type{ChainState}
      & (\Block^?,~\Slot,~\UTxO)
      & \text{Chain state} \\
    \end{array}
  \end{equation*}
  \caption{Types of a block-based ledger}
  \label{fig:block-thread}
\end{figure*}

In blockchains, as the name suggests, the atomic state update
is done by a block rather than a transaction. Blocks contain lists of transactions, as well
as other data used to perform the update.
Block application updates the \emph{chain state}, which the state updated by
transactions is a part of. So, blocks play the role of the $\Tx~\leteq~\Block$ type in our
model, and the $\State$ refers to the chain state.

We define the initial state by $\fun{initState}~\leteq~(\Nt,~0,~\empty)$, and the block application function is

\[ \ups~(b,~lstx)~(b_c,~slot,~utxo)~=~\]

\[\begin{cases}
  \Err & \text{ if } \neg~\fun{blockChecks}~(b,~lstx)~(b_c,~slot,~utxo) \\
  ((b,~lstx),~slot~+~1,~\fun{foldl}~(\fun{updateUTxO}~slot)~utxo~lstx) & \text{ otherwise }
\end{cases} \]

Performs the following checks via the $\fun{blockChecks}$ function, and returns $\Err$
either fails :

\begin{itemize}
  \item[(i)] $b~==~b_c$, which ensures that the (last) block recorded in the state is
  the same block to which the new block specifies that it must be attached

  \item[(ii)] $lstx~\neq~[]$, which ensures that blocks must contain at least one
  transaction

  \item[(ii)] $\fun{foldl}~(\fun{updateUTxO}~slot)~utxo~lstx~\neq~\Err$, which verifies
  that the UTxO set is updated without errors by left-folding $\fun{updateUTxO}$ over
  the list of transactions.
\end{itemize}

The $\fun{updateUTxO}$ update is done in the usual way, by removing all entries corresponding
to transaction inputs, and adding ones corresponding to the outputs. The following
checks are performed, and $\Err$ is returned by $\fun{updateUTxO}~slot~(lsin,~outs,~lastslot)~utxo$
if any of them fail :

\begin{itemize}
  \item[(i)]  $lsin ~\neq~ \Nt$ ensures the transaction has at least one input  \\

  \item[(ii)] ensures the current slot is before or equal to the slot up to which the
  transaction is valid

  \item[(iii)] $\forall i \in lsin, i \mapsto \wcard \in utxo$ ensures all the transaction
  inputs correspond to existing UTxO entries \\

  \item[(iv)] $~\Sigma_{\wcard \mapsto v\in outs} v~ \neq~\Sigma_{v\in \{v~\mid~i \in lsin \wedge  i\mapsto v~\in~utxo \}} v$
  is the preservation of value condition, which ensure that the total value in the consumed UTxO entries
  is equal to the value in the entries produced  \\

\end{itemize}

\[ \fun{updateUTxO}~slot~(lsin,~outs,~lastslot)~utxo~=~\]

\[  (utxo~setminus~\{ i \mapsto \wcard \mid i \in lsin \}) \cup \{ ((lsin,~outs),~ix) \mapsto v \mid ix \mapsto v \in outs ... \}  \]

Note here that we use the full block and transaction data where their hashes would
normally be used instead : within keys of UTxO entries, and inside
blocks to specify the preceeding block. The reason for this is that properties
we discuss here may not hold up without the assumption there will be no hash collisions,
and additional machinery is required to reason under this assumption, while using
hashes instead of preimages does not appear to provide additional insight into the
properties we study here. However, a closer examination of this claim could be
part of future work.

We see immediately that the update function of the UTxO set (along with the
constraints on it) in this example can be specified by

\[ \fun{updateUTxOThread} ~lstx ~(\wcard,~slot,~ utxo)~ utxo'~ \leteq~ \begin{cases}
  \fun{foldl}~(\fun{updateUTxO}~slot)~utxo'~lstx & \text { if } utxo~==~utxo' \\
  \Err & \text{ otherwise}
\end{cases}  \]

Since the thread update is only valid if the argument corresponding to the state of the
thread (here, $utxo'$) is equal to the the projection from the chain state to the
thread state (here, $utxo$ in the chain state triple).

The slot number is not used in the generation
of the updated UTxO set in the case when a UTxO set, and not $\Err$, is returned. However, in order to
determine whether a thread update is permitted, the $slot$ value, which
is not included in the UTxO thread data, is used needed.
We will show that it is in fact a constraint on the design of deterministic ledgers
that this pattern of thread update
be followed, ie. the updated state of a thread can depend on data external to the thread,
but only in so far that the update can produce an error for certain values if that data.

We can make this example more general by remaining agnostic of the data types of the
chain state. Following this example,

\begin{definition}
  \begin{itemize}
    \item[(i)] Given a type $\Tx$, a $\Block$ is a type which admits
    the following accessors :

    \[ \fun{prevBlock} \in \Block \to \Block \]

    \[ \fun{txList} \in~\Block \to [\Tx] \]

%    \[ \fun{time} \in \type{Header} \to \Time \]

    \item[(ii)] A $\type{ChainState}$ is a type with the following accessors

    \[ \fun{lastBlock} \in \type{ChainState}~\to \Block \]

    \[ \fun{txState} \in \Block \to \type{TxState}  \]

    where $\type{TxState}$.

    \item[(iv)] Given an initial state $\fun{initState}~\in~\Block$ and an update procedure
    $\ups~\in~\Block~\to~\type{ChainState}~\to~\type{ChainState}$, the tuple
    $(\Block,~\type{ChainState},~\ups,~\fun{initState})$ is said to be a
    \emph{blockchain} if

    \begin{itemize}
      \item[(1)] it is a ledger specification,

      \item[(2)] it admits a thread $\type{TxState}$ with the update function defined by

     \[ \fun{updateLedger} ~lstx ~(led,~ os)~ led'~ \leteq~ \begin{cases}
       \fun{foldl}~(\fun{updateLedger}~os)~led'~lstx & \text { if } led~==~led' \\
       \Err & \text{ otherwise}
     \end{cases}  \]

    \end{itemize}

     and

    wherein $(\fun{updateLedger}~os)$

    A  $(\Block,~\type{ChainState},~\fun{initState})$
    and an update procedure


  \end{itemize}

  , as we will see in the next definition, represents the
  part of the $\type{ChainState}$ that applying transactions updates

\end{definition}

Recall here, again, that our formalism is not probabilistic, so, unlike a realistic
deployed blockchain, it does not rely on hash functions having a low probability
of collisions. Instead, we use pre-images where hashes would normally be used in
practice (eg. as pointers to previous blocks).

we will refer to the \emph{ledger} as the thread in the
chain state that can be described as \emph{any data updatable by a transaction}. Formally,

While the above specification is a fairly abstract idea of what a blockchain is,
some distributed-consensus append-only ledgers may have a structure which deviates from the
specification we present here. However, the
\cite{Cardano-ledger-spec} \cite{tezos} \cite{ethereum} \cite{Nakamoto}



In the case of updating the UTxO set and the protocol parameter record, this
situation can manifest itself as, eg. a specifying a UTxO update which disallows some
new UTxOs to be added to it if they do not satisfy a constraint that depends on
one of the protocol parameters. There might be, for instance, a "minimum value constraint"
imposed by the $\fun{updateT}_{UTxO}~tx$ function
on any new UTxO being added by a $tx$ such that the value contained in a UTxO entry must be greater
than a value indicated in some protocol parameter in $\fun{minV}$. The update specification
for protocol paramenters may also have some constraints.

Any transaction that is valid (ie. does not produce an $\Err$) must satisfy both the
constraints on the UTxO update, as well as those for the protocol parameter updates.


\bibliographystyle{splncs04}
\bibliography{abstract-ledgers}

\end{document}
